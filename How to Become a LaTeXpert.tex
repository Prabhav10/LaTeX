% PREAMBLE %

\documentclass[12pt]{report}

% Packages %
\usepackage[margin=3.5cm, includefoot]{geometry} % Margins %
\usepackage{dirtytalk} % Quotations %
\usepackage{fancyhdr} % Headers + Footers %
\usepackage{shapepar} % Paragraph Shaping %
\usepackage{verse} % Poetry %
\usepackage[toc]{appendix}
\usepackage{imakeidx} % Index %
\usepackage{xcolor} % Colors for Headings %
\usepackage{sectsty} % Changes Headings %
\usepackage{fontspec} % Installing other fonts %
\usepackage{tcolorbox} % Colored Boxes %
\usepackage{float} % Table Positioning %
\usepackage{multirow} % Multiple Cells over Rows in Table %
\usepackage{tikz} % Drawings % 
\usepackage{circuitikz} % Circuits %
\usepackage{pgfplots} % Graphs %
\usepackage{graphicx} % Graphics %
\usepackage{mhchem} % Chemistry % 
\usepackage{chemfig} % Drawing Chemistry %
\usepackage{showexpl} % LaTeX Code + Output %
\usepackage{listings} % Code Listing %
\usepackage{fancyvrb} % LaTeX Code + No Output %
\usepackage[colorlinks, citecolor=green, urlcolor=blue, linkcolor=magenta, hypertexnames=true]{hyperref}  % Hyperlinks %
\usepackage{mathtools, amssymb, amsthm, braket, polynom, cancel} % Math %
\usepackage[ISO]{diffcoeff} % Derivatives %
\usepackage{polyglossia} % International Language Support %
\setdefaultlanguage{english} % Main Language %
\setotherlanguages{spanish, french} % Other Languages %
\usepackage[UTF8, scheme = plain]{ctex} % Chinese Text %
\usepackage{xeCJK} % Japanese Text %
\usepackage{metalogo} % XeLaTeX %

% Commands %

% Index Command
\newcommand{\indexcommand}[1]{\index{#1@$\texttt{\textbackslash#1}$}\index{commands!$\texttt{\textbackslash#1}$}} 
% Print + Index Command 
\newcommand{\command}[1]{\texttt{\textbackslash#1}\index{#1@$\texttt{\textbackslash#1}$}\index{commands!$\texttt{\textbackslash#1}$}} 
% Index Font Commands 
\newcommand{\fontCommand}[2]{\index{#1@$\texttt{\textbackslash#1}$}\index{commands!$\texttt{\textbackslash#2}$@$\texttt{\textbackslash#1}$}\index{font sizes!$\texttt{\textbackslash#2}$@$\texttt{\textbackslash#1}$}}


% Math %
\newcommand{\dee}{\mathrm{d}} % Upright d 
\newcommand{\bvec}[1]{\boldsymbol{#1}} % Bold Vector 
\newtheoremstyle{dotless}{}{}{}{}{\bfseries}{}{12pt}{}
\theoremstyle{dotless}
\newtheorem{thm}{Theorem}[section] % Theorem 
\newtheorem*{thm*}{Theorem} % Theorem (unnumbered)
\newtheorem{lem}[thm]{Lemma} % Lemma 
\newtheorem*{lem*}{Lemma} % Lemma (unnumbered)
\newtheorem{cor}[thm]{Corollary} % Corollary 
\newtheorem*{cor*}{Corollary} % Corollary (unnumbered)
\newtheorem{prop}[thm]{Proposition} % Proposition
\newtheorem*{prop*}{Proposition} % Proposition (unnumbered)
\newtheorem{defn}[thm]{Definition} % Definition
\newtheorem*{defn*}{Definition} % Definition (unnumbered)
\newtheorem{rem}[thm]{Remark} % Remark
\newtheorem*{rem*}{Remark} % Remark (unnumbered)
\newtheorem{exa}[thm]{Example} % Example
\newtheorem*{exa*}{Example} % Example (unnumbered)
\newtheorem{exe}[thm]{Exercise} % Exercise
\newtheorem*{exe*}{Exercise} % Exercise (unnumbered)
\newtheorem{aside}[thm]{Aside} % Aside
\newtheorem*{aside*}{Aside} % Aside (unnumbered)

% Fonts %
\renewcommand{\familydefault}{\sfdefault} % sets document fonts
\newfontfamily{\tnr}{Times New Roman} % Times New Roman
 \let\arabicfonttt\ttfamily
\newfontfamily{\arabicfont}[Script=Arabic,Scale=1.1]{Scheherazade} % Arabic Font 
\newfontfamily{\hindifont}[Script=Devanagari]{Lohit Devanagari} % Hindi Font 
\newfontfamily{\bengalifont}[Script=Bengali]{Kalpurush} % Bengali Font
\newfontfamily{\russianfont}[Script=Cyrillic]{Linux Libertine} % Russian Font 
\newfontfamily{\thaifont}[Script=Thai]{Prompt} % Thai Font 
\newfontfamily{\hebrewfont}[Script=Hebrew]{IBM Plex Sans Hebrew} % Hebrew Font
\newfontfamily{\greekfont}[Script=Greek]{Linux Libertine} % Greek Font 

% Colours %
\definecolor{deepjunglegreen}{rgb}{0.0, 0.29, 0.29} % define my green color
\sectionfont{\color{deepjunglegreen}}  % sets colour of sections
\subsectionfont{\color{deepjunglegreen}}  % sets colour of sections

% Options on Headers + Footers %
\pagestyle{fancy}
\fancyhf{}
\chead{\leftmark}
\cfoot{\thepage}
\usepackage[bottom]{footmisc} % Places footnote at bottom of the page 
\renewcommand{\chaptermark}[1]{ % Chapter Mark %
                 \markboth{\large\textbf{#1}}{}}

% Alphabetical Index Options %
\begin{filecontents*}{\jobname.mst}
headings_flag 1
heading_prefix   "\\par\\penalty-50\\textbf{"
heading_suffix   "}\\\\\*\~\\\\\*"
symhead_positive "Symbols"
symhead_negative "symbols"
numhead_positive "Numbers"
numhead_negative "numbers"
delim_0 ",\~"
\end{filecontents*}

% An Index %
\makeindex[intoc]

% Options on Code + Output %
\lstdefinestyle{myLaTeX}
{
    language=[LaTeX]TeX,
    morekeywords={say},
    keywordstyle=\ttfamily,
    numbers=none,
    basicstyle=\footnotesize\ttfamily,
    breakatwhitespace=true,         
    breaklines=true,
    captionpos=b,               
    keepspaces=true,                             
    showspaces=false,                
    showstringspaces=false,
    showtabs=false,                  
    tabsize=2,
}

% Spacing %
\frenchspacing % Treats punctuation equally %

% Document Info %
\title{How to Become a \LaTeX pert}
\author{\color{deepjunglegreen}Prabhav Kumar}
\date{\color{deepjunglegreen}\today}

% PREAMBLE END %

% DOCUMENT %
\begin{document}

% Title Page %
\maketitle

% Table of Contents %
\tableofcontents

% Setting Page Number %
\setcounter{page}{6}

% CHAPTER 0 %
\chapter{A Very Brief Introduction}
This guide serves as an introduction to \LaTeX{}. I have tried to make it concise and easy to follow. I am open to criticism (hopefully constructive). \\\\ Contact me via \url{prabhavkumar10@gmail.com} or \href{https://www.linkedin.com/in/prabhavkumar/}{LinkedIn}. \\\\ Updates to this guide can be found on \href{https://github.com/Prabhav10/LaTeX}{GitHub}.

% CHAPTER 1 %

\chapter{\LaTeX{} 101}

% TeX %
\section{\TeX{}}
Throughout history, mathematical symbols and equations were written on materials such as clay and paper. With the advent of technology, humans needed a way to write math on computers, so \TeX{} (pronounced \say{Tech}) was created. It is a computer program created by Donald E. Knuth\index{Donald E. Knuth} and short for {\greekfont{τέχνη}}, Greek for both \say{art} and \say{craft}.

% LaTeX %
\section{\LaTeX{}}
\LaTeX{} (pronounced \say{Lah-Tech}) is built on top of \TeX{} and is much more user-friendly. It is created by Leslie Lamport\index{Leslie Lamport}. It is useful to think of \TeX{} as a low-level language and \LaTeX{} as a higher-level language. 

% Analogy %
\subsection{The Light-Bulb Analogy}
Understanding the difference between \TeX{} and \LaTeX{} can be quite tricky, so a simple light-bulb analogy can be used to explain the difference: \\\\ \TeX{} and \LaTeX{} can be thought of an electrical circuit and a switch, respectively. Both provide a similar function (lighting up a bulb), but it is much more convenient for the user to deal with the switch than the circuit.

% Word %
\subsection{Not a Word Processor}
\LaTeX{} is \textbf{not} a word processor. When using Microsoft Word\index{Microsoft Word}, the final document automatically updates while typing on a .docx file\index{file}. However, with \LaTeX{}, the final document, which is usually a separate PDF file, only updates after typing and typesetting or compiling a .tex file\index{file}. 

% Installation %
\section{Installing \LaTeX{}}
There are several ways to install\index{install} \LaTeX{}, but this is what I did:
\begin{enumerate} 
	\item Download \href{https://miktex.org/download}{MikTeX} (a \TeX{} distribution) . 
	\item Download \href{https://www.xm1math.net/texmaker/download.html}{Texmaker} (a cross-platform \LaTeX{} editor). 
	\item Watch \href{https://www.youtube.com/watch?v=WnIYTFTsWiU}{this} video for a guided walkthrough.
\end{enumerate} 

% CHAPTER 2 %

\chapter{Creating your First \LaTeX{} Document}

% Pre-requisites, Typesetting & Troubleshooting
\section{Before you Start: Typesetting \& Troubleshooting\index{typesetting}\index{troubleshooting}\index{start}}
To compile a \LaTeX{} file, the user must typeset it on Texmaker using the \say{Quick Build} button (other editors have a \say{Typeset} or \say{Compile} button). If a PDF\index{PDF} file is outputted, the compilation\index{compilation} was successful. If not, then do the following:\\ 
\begin{enumerate}
\item Click on the "abort" button.
\item Read the console output - it will include the line number and the command that caused the error.
\item Fix the error\index{error}.
\item Typeset the file again. \\
\end{enumerate}
As an example, after opening a new document, type the following code:
\begin{LTXexample}[style=myLaTeX, pos=o]
\documentclass[10pt]{article}

\begin{document}
Hello World!
\end{document}
\end{LTXexample}
As shown, the output should be Hello Word!.
\index{environments!document@\texttt{document}}


% Document Class %
\section{\command{documentclass}}
\label{classes}
\command{documentclass} is the command that must appear at the start of a \LaTeX{} document. The document class is specified within \{ \}. Frequently used classes include: 
\begin{itemize}
\item \command{article} - for shorter documents (notes).
\item \command{beamer} - for presentation slides.
\item \command{book} - self-explanatory.
\item \command{proc} - for conference proceedings.
\item \command{report} - for longer documents (PhD thesis\index{thesis}). \\
\end{itemize} 
Document class options are specified within [ ]. Frequently used options include: 
\begin{itemize}
\item \texttt{$x$pt\index{font}} - main font size (default is 10pt).
\item \texttt{a4paper\index{A4 paper}}, \texttt{a5paper\index{A5 paper}}, \texttt{letterpaper\index{Letter paper}}, \texttt{legalpaper\index{Legal paper}} - paper size. 
\end{itemize}

% Preamble %
\section{The Preamble\index{preamble} \& The Body\index{preamble}}
Anything before \command{begin}\texttt{\{document\}} is called the preamble and applies to the whole document. \\\\
The area between \command{begin}\texttt{\{document\}} and \command{end}\texttt{\{document\}} is called the body. This is where the content goes. \\\\
Text after \command{end}\texttt{\{document\}} will be ignored.

% Packages %
\subsection{\command{usepackage}}
Sometimes \LaTeX{} cannot solve a problem, so external packages are added in the preamble using \command{usepackage}. For example, \texttt{dirtytalk}\index{quotations} allows users to deal with quotation marks.
\begin{LTXexample}[style=myLaTeX, pos=o]
\documentclass[10pt]{article}
\usepackage{dirtytalk}

\begin{document}
\say{Stay positive \& test negative!}
\end{document}
\end{LTXexample}
$\\$
You can find package documentation by googling \say{CTAN *insert package name*}.

% Title / Author / Date %
\subsection{\command{title}, \command{author}, \command{date}, \command{maketitle}}
When creating a title page, these 4 commands are used. For example, if Einstein (hypothetically) wrote a \LaTeX{} document, he would type \command{title}\texttt{\{General Relativity\}}, \command{author}\texttt{\{Albert Einstein\}}, \command{date}\texttt{\{May 7, 1915\}} and \command{maketitle} in the body.

% Special Characters %
\section{The Building Blocks of \LaTeX{}: \texttt{\textbackslash}, \texttt{[  ]}, \texttt{\{ \}}, \texttt{\textvisiblespace}}
The backslash is central to \LaTeX{} because each \LaTeX{} command starts with \texttt{\textbackslash}. 

% Commands %
\subsection{Commands\index{commands}}
A command is a special expression that instructs \LaTeX{} to do a specific task. It is case-sensitive, i.e. \texttt{\textbackslash large}\fontCommand{large}{large} and \texttt{\textbackslash Large}\fontCommand{Large}{large} are different commands. Commands are sometimes followed by declarations. 

% Declarations %
\subsection{Declarations\index{declarations}}
A declaration is either optional \texttt{[  ]} or required \texttt{\{ \}}. For example, in \\ \command{documentclass}\texttt{[10pt]\{article\}}:
\begin{itemize}
\item \command{documentclass} is the command.
\item \texttt{[10pt]} is an optional declaration (can be omitted).
\item \texttt{\{article\}} is a required declaration (something is needed within \texttt{\{ \}}). \\
\end{itemize} 
Sometimes \texttt{\{ \}} provides spacing after a command:
\begin{LTXexample}[style=myLaTeX, pos=o]
\LaTeX  ample vs. \LaTeX{} ample
\end{LTXexample}

% Environments %
\subsection{Environments\index{environments}}
An environment performs a specific action on a block of \LaTeX{} code. It must have matching \command{begin} and \command{end} declarations. For example:
\begin{LTXexample}[style=myLaTeX, pos=o]
\begin{center}
Core-an is the official language at the center of Earth. \\
P.S. This is how you center text.
\end{center}
\end{LTXexample}
\index{center}
\index{center text}
\index{justify center}
\index{environments!center@\texttt{center}}

% Spaces %
\subsection{Spaces\index{spaces}}
Spaces and tabs\index{tabs} are treated as 1 space (\texttt{\textvisiblespace}). Any combination of consecutive spaces and tabs are also treated as 1 space. An empty line\index{empty line} marks the end of a paragraph\index{paragraph}. Consecutive empty lines are treated  as 1 empty line. For example:
\begin{LTXexample}[style=myLaTeX, pos=o]
1 space =    1 tab.
1 space =      consecutive spaces. 

1 empty line = end of paragraph.


1+ empty lines = 1 empty line.
\end{LTXexample}
$\\$
Use a tilde (\~{}) for an unbreakable space
\begin{LTXexample}[style=myLaTeX, pos=o]
This is an unbreakable space.
Really? Oh Yes. \\
This is an unbreakable space.
Really? Oh~Yes. 
\end{LTXexample}
\index{unbreakable space}

% Special Characters %
\subsection{Special Characters\index{special characters}}

% Reserved Characters %
\subsubsection{Reserved Characters\index{reserved characters}}
The following characters\index{characters} (discussed elsewhere) have a special meaning in \LaTeX: 
\begin{center}
  {\verb! # $ % & { } _ ~ ^ \!}
\end{center}
$\\$
In order to print them, you must escape or prefix the character with a \texttt{\textbackslash}: 
\begin{LTXexample}[style=myLaTeX, pos=o]
\# \$ \% \& \{ \} \_ \~{} \^{}
\textbackslash
\end{LTXexample}
$\\$
\command{\textbackslash} means a newline\index{newline} so \texttt{\textbackslash textbackslash} is used.
\indexcommand{\$}
\index{dollar sign}
\indexcommand{\%}
\index{percent sign}
\index{comment}
\index{escape}

% Symbols %
\label{symbols}
\subsubsection{Symbols\index{symbols}}
There are over 18,000 symbols\footnote{Some symbols require packages. } in \LaTeX{}. They are printed using specific commands\index{commands}. For example: 
\begin{LTXexample}[style=myLaTeX, pos=o]
$\div$ \\
\copyright \\
$\clubsuit$ 
\end{LTXexample}
$\\$
Sometimes the symbol must be placed between \$ signs (refer to \ref{math mode} on page \pageref{math mode}).
\index{$\clubsuit$} \indexcommand{clubsuit}
\index{\copyright} \indexcommand{copyright}
\index{$\div$} \indexcommand{div}\index{divide}\index{division}

% Symbols Resources %
\subsubsection{Helpful Resources\index{resources}}
\begin{enumerate}
\item \href{https://detexify.kirelabs.org/classify.html}{Detexify} - inputs a drawing of a symbol and outputs its \LaTeX{} command (and the package required if needed).
\item \href{http://tug.ctan.org/info/symbols/comprehensive/symbols-letter.pdf}{CTAN} - a comprehesive list of symbols.
\end{enumerate}

% Organization %
\section{Basic Organization (Section)\index{organization}\index{sections}}
It is helpful to divide the \LaTeX{} document into different sections.

% Sectioning Commands %
\subsection{Sectioning Commands\index{sectioning commands} (Subsection)}
Different document classes\index{document classes} have different sectioning commands\index{commands}:

% Article Sectioning Commands %
\subsubsection{\texttt{article} class (Subsubsection)}
\begin{itemize}
\item \command{section}
\item \command{subsection}
\item \command{subsubsection}
\item \command{paragraph}
\end{itemize}

% Book/Report Sectioning Commands %
\subsubsection{\texttt{book, report} class (Subsubsection)}
\begin{itemize}
\item \command{part}
\item \command{chapter}
\item All of the \texttt{article} sectioning commands.
\end{itemize}
$\\$
Sectioning commands automatically provide spacing\index{space} (and numbering), so you do not need to add \command{newline} or \command{\textbackslash} before the next sectioning command.  Add a $*$ after the sectioning command for unnumbering. (Paragraph)
\begin{lstlisting}[style=myLaTeX, columns=fullflexible]
\section{This will be a numbered section.}
\section*{This will not.}
\end{lstlisting} 

% Labelling %
\subsection{Labelling\index{labels}\index{references}}
\label{label}
Section commands can be labelled with \command{label}\texttt{\{labelname\}}. When referring to a particular section, use  \command{ref}\texttt{\{labelname\}} (the section) or \command{pageref}\texttt{\{labelname\}} (page number of the section). Using an example from this guide:
\begin{LTXexample}[style=myLaTeX, pos=o]
\label{symbols}
% Some Code 
To learn about symbols, refer to \ref{symbols} on page \pageref{symbols}.
\end{LTXexample}

% Footnotes %
\subsection{Footnotes \index{size}\index{footnotes}}
\command{footnote} prints text at the bottom of the page\footnote{Footnotes are easy with \LaTeX{}!}. Here is the code I used to create the footnote:
\begin{LTXexample}[style=myLaTeX, pos=o]
$\ldots$ at the bottom of the page\footnote{Footnotes are easy with \LaTeX{}!}. 
\end{LTXexample}

% Comments %
\subsection{Comments\index{comments}}
The \texttt{\%}\indexcommand{\%}\index{percent sign} character is reserved for commenting. When used, the rest of the current line is ignored.
\begin{LTXexample}[style=myLaTeX, pos=o]
% This text will be ignored.
Some text % will be ignored.
\end{LTXexample}
$\\$
If multiple lines need to be commented, highlight the necessary text and use the keyboard shortcut for commenting\footnote{Keyboard shortcuts for \href{https://www.xm1math.net/texmaker/doc.html\#SECTION39}{Texmaker}; \href{http://transit.iut2.upmf-grenoble.fr/doc/texstudio/html/usermanual_en.html\#SHORTCUTS}{Texstudio}.}\index{keyboard shortcuts}. \\\\ Another way is to define a new command:
\begin{lstlisting}[style=myLaTeX, columns=fullflexible]
% Preamble
\newcommand{\comment}[1]{}

% Body
\comment{
You Can't See Me.

Just Like John Cena.
}
\end{lstlisting} 

 % CHAPTER 3 %

\chapter{All About Text}

% Line Breaking %
\section{Line Breaking\index{line breaks}\index{newline}}
As mentioned previously, \command{\textbackslash} or \command{newline} performs a line break. \command{\textbackslash$*$} starts a new line without starting a new paragraph. 
\begin{LTXexample}[style=myLaTeX, pos=o]
Text is broken here. \\* 
The paragraph is not broken.
\end{LTXexample}

% Page Breaking %
\section{Page Breaking\index{page breaks}\index{new page}}
\command{newpage} should suffice.

% Keeping Words Together %
\section{Keeping Words Together\index{coalescing words}}
\command{mbox} keeps words together in 1 line. No line breaks are allowed in the text.
\begin{LTXexample}[style=myLaTeX, pos=o]
These words are not grouped, but \mbox{these words are}.
\end{LTXexample}
$\\$
\command{fbox} draws a box around the grouped words.
\begin{LTXexample}[style=myLaTeX, pos=o]
\fbox{These words are trapped}.
\end{LTXexample}

% Punctuation %
\section{Punctuation\index{punctuation}}

% Type-able punctuation %
\subsection{Apostrophes, Colons, Commas, Periods, and Semi-Colons}
Simply type these punctuation marks. It is useful but not necessary to type \command{frenchspacing} in the preamble. This tells the document to treat spacing after commas and periods equally. 

% Dashes %
\subsection{Dashes\index{dashes}\index{hyphens}\index{en-dash}\index{em-dash}\index{minus sign}}
There are 4 types of dashes:
\begin{LTXexample}[style=myLaTeX, pos=o]
hyphen (-): Pre-Christmas vibes. \\
en-dash (--): I work from 9--5. \\
em-dash (---): Yes --- or no? \\
minus sign ($-$): $-69$.
\end{LTXexample}

% Ellipsis %
\subsection{Ellipsis\index{ellipsis}\index{dots}}
As it has better spacing and line-break behavior, \command{ldots} in between dollar signs is a better solution than typing 3 dots.
\begin{LTXexample}[style=myLaTeX, pos=o]
3 dots:  a b c... z \\
low dots: a b c $\ldots$ z
\end{LTXexample}
$\\$


% Quotation Marks %
\subsection{Quotation Marks\index{quotation marks}}
The \texttt{dirtytalk} package is a comprehensive solution.
\begin{LTXexample}[style=myLaTeX, pos=o]
% Preamble
\usepackage{dirtytalk}

% Body
\say{I am surrounded!} \\
\say{Quotations can be \say{nested} as well!}
\end{LTXexample}
$\\$
If you want to define the primary and secondary set of quotation marks, type the following in the preamble:
\begin{lstlisting}[style=myLaTeX, columns=fullflexible]
\usepackage[
    left = ``,%
    right = '',% 
    leftsub = `,% 
    rightsub = ',%
]{dirtytalk}
\end{lstlisting}
$\\$
You can find a table of primary and secondary quotation marks in several languages \href{https://www.overleaf.com/learn/latex/Typesetting_quotations#Reference_guide}{here}. 

% Quotations %
\subsection{Quotations\index{quotations}}
The \texttt{quotation} environment\index{environments!quotation@\texttt{quotation}} adds quotes to some text.
\begin{LTXexample}[style=myLaTeX, pos=o]
Walt Disney once said something interesting.
\begin{quotation}
The way to get started is to quit talking and begin doing. 
\end{quotation}
It changed my life.
\end{LTXexample}

% More Special Characters %
\section{More Special Characters\index{special characters}\index{characters}}
Commands for accents, diacritics, and other characters can be found on \href{https://en.wikibooks.org/wiki/LaTeX/Special_Characters}{Wikibooks}. These commands may not be needed if the characters can be typed out from the keyboard.
\index{accents}\index{diacritics}\index{symbol}

% Aligning Text %
\section{Aligning Text\index{text-align}\index{align}}
The \texttt{flushleft}, \texttt{center}, \texttt{flushright} environments align text to the left, center, and right, respectively.

\begin{LTXexample}[style=myLaTeX, pos=o]
\begin{flushleft}
Chairman Mao
\end{flushleft}

\begin{center}
Canada
\end{center}

\begin{flushright}
Hitler
\end{flushright}
\end{LTXexample}
\index{environments!flushright@\texttt{flushright}}
\index{environments!flushleft@\texttt{flushleft}}
\index{environments!center@\texttt{center}}
\index{align left}
\index{align right}
\index{align center}
\index{center}
\index{justify left}
\index{justify center}
\index{justify right}

% Coloring Text %
\section{Coloring Text\index{color}\index{text-color}}
\label{colors}
Add color to the document using the \texttt{xcolor} package. Use \command{textcolor} or \command{color} to color text.
\begin{LTXexample}[style=myLaTeX, pos=o]
% Preamble
\usepackage{xcolor}

% Body
\textcolor{yellow}{Black} and {\color{black}yellow}.
\end{LTXexample}
$\\$
If you want to highlight text\index{text-highlight}\index{highlighting}, then use \command{colorbox}.
\begin{LTXexample}[style=myLaTeX, pos=o]
% Preamble
\usepackage{xcolor}

% Body
\colorbox{yellow}{Black} \\
\colorbox{black}{\textcolor{yellow}{yellow}}
\end{LTXexample}
$\\$
The colors provided by \texttt{xcolor} can be found \href{https://www.overleaf.com/learn/latex/Using_colours_in_LaTeX\#Named_colours_provided_by_the_xcolor_package}{here}. If you scroll down on the webpage, there are instructions to define more colors and set the page color.

% Fonts %
\section{Fonts\index{fonts}}

% Font Sizes %
\subsection{Font Sizes \index{size}\index{font sizes}}
\LaTeX{} commands for font sizes:
\begin{LTXexample}[style=myLaTeX, pos=o]
{\tiny tiny} \\
{\scriptsize scriptsize} \\
{\footnotesize footnotesize} \\
{\small small} \\
{\normalsize normalsize} \\
{\large large} \\
{\Large Large} \\
{\LARGE LARGE} \\
{\huge huge} 
\end{LTXexample} 
\fontCommand{tiny}{tiny}
\fontCommand{scriptsize}{scriptsize}
\fontCommand{footnotesize}{footnotesize}
\fontCommand{small}{small}
\fontCommand{normalsize}{normalsize}
\fontCommand{large}{large}
\fontCommand{Large}{large}
\fontCommand{LARGE}{large}
$\\$
A more thorough tutorial on font sizes can be found at \href{https://latex-tutorial.com/changing-font-size/}{latex-tutorial.com}.

% Typefacing %
\subsection{Font Styles\index{typeface}\index{font styles}}
Text can be type-faced in different ways. Popular commands include: 
\begin{LTXexample}[style=myLaTeX, pos=o]
\underline{Underlined text} \\
\emph{Emphasized text.} \\
\textbf{Bold text.} \\
\textit{Italicized text.} \\
\textsl{Slanted text.} \\
\textsc{Smallcaps text.} \\
\uppercase{upper.} \\
\lowercase{LOWER.} 
\end{LTXexample} 
$\\$
\command{underline} does not break properly to the next line, so use the \href{https://ctan.org/pkg/ulem?lang=en}{\texttt{ulem}} pacakge to resolve the issue. This package also allows for underline styling and strikethroughs\index{strikethrough}. \\\\ Note that although \command{emph}, \command{textit}, and \command{textsl} have the same effect here, they render different effects with other fonts.\indexcommand{underline}\index{underline}
\indexcommand{emph}\index{emphasize}
\indexcommand{textbf}\index{bold}
\indexcommand{textit}\index{italicize}
\indexcommand{textsl}\index{slanted}
\indexcommand{textsc}\index{smallcaps}
\indexcommand{uppercase}\index{uppercase}
\indexcommand{lowercase}\index{lowercase}

% Font Families %
\subsection{Font Families \index{font families}\index{typefaces}}
The default font on a \LaTeX{} document is \textrm{Computer Modern}\index{Computer Modern@\textrm{Computer Modern}}, which is part of the \textrm{serif}\index{serif@\textrm{serif}} family. The other 2 popular font families are \textsf{sans serif}\index{sans serif} and \texttt{monospace}\index{monospace@\texttt{monospace}}\index{typewriter@\texttt{typewriter}}. 
\index{font families!monospace@\texttt{monospace}}
\index{font families!serif@\textrm{serif}}
\index{font families!sans serif}

% Global Font Family % 
\subsubsection{Changing the Global Font Family\index{global font}\index{permanent font}}
To change the default font family for the whole document to \texttt{monospace}, type the following in the preamble: 
\begin{lstlisting}[style=myLaTeX, columns=fullflexible]
\renewcommand{\familydefault}{\ttdefault}
\end{lstlisting} 
$\\$
Replace \command{ttdefault} with \command{sfdefault} for sans serif. As \textrm{serif} is the default font family for a \LaTeX{} document, nothing needs to be done to use it. However, if needed, use \command{rmdefault}.

% Local Font Family % 
\subsubsection{Temporarily Changing the Font Family\index{local font}\index{temporary font}}
To use these font families temporarily, use the following commands:
\begin{LTXexample}[style=myLaTeX, pos=o]
\textrm{Serif text.} \\
\textsf{Sans Serif text.} \\
\texttt{Monospace text.}
\end{LTXexample} 
\indexcommand{textrm}
\indexcommand{textsf}
\indexcommand{texttt}

% More Fonts %
\section{More Fonts\index{external fonts}}

To use other fonts, employ external packages.

% Fonts installed on your PC %
\subsection{Using Other Fonts Installed on your PC\index{PC fonts}\index{word processor fonts}\index{other fonts}}

% Fonts installed on your PC (globally) %
\subsubsection{Using the font globally\index{global font}}
To use fonts installed on a local PC (e.g.{\tnr Times New Roman}), type the following in the preamble:
\begin{lstlisting}[style=myLaTeX, columns=fullflexible]
\usepackage{fontspec}
\setmainfont{Times New Roman}
\end{lstlisting}
$\\$
Then, you must typeset the document using the \XeLaTeX{}\index{XeLaTeX@\XeLaTeX} or \LuaLaTeX\index{LuaLaTeX@\LuaLaTeX} compiler\index{compiler}. The section between 2:52 to 3:16 of \href{https://www.youtube.com/watch?v=oI8W4MvFo1M&t=2m52s}{this} video might help with the compilation.

% Fonts installed on your PC (temporary) %
\subsubsection{Using the font temporarily\index{temporary font}}
If you only need to use {\tnr Times New Roman} for some text, type:
\begin{LTXexample}[style=myLaTeX, pos=o]
\usepackage{fontspec}
\newfontfamily{\tnrm}{Times New Roman} 
{\tnrm Times New Roman Text.}
\end{LTXexample} 
$\\$
\texttt{\textbackslash tnrm} was my choice. It can be replaced with a command of your choice (just make sure it isn't already defined). \\\\
The user can download fonts for their PC at \href{https://www.dafont.com}{dafont.com}.

% Font Catalogue %
\subsection{Using The \LaTeX{} Font Catalogue\index{font catalogue}}
\label{mathfont}

More fonts and their instructions-for-use can be found on \href{https://tug.org/FontCatalogue/}{The \LaTeX{} Font Catalogue}\footnote{If you are writing mathematical expressions, use fonts with math support.}. For example, if you want to use {\fontfamily{pbk}\selectfont Bookman}\index{bookman@{\fontfamily{pbk}\selectfont Bookman}}\footnote{Popular fonts can be found \href{https://r2src.github.io/top10fonts/}{here}.} as the global font, type the following in the preamble: 
\begin{lstlisting}[style=myLaTeX, columns=fullflexible]
\usepackage{bookman}
\end{lstlisting} 
$\\$
If you want to use {\fontfamily{pbk}\selectfont Bookman} only for some text:
\begin{LTXexample}[style=myLaTeX, pos=o]
Normal Text. \\
{\fontfamily{pbk}\selectfont Bookman Text.}
\end{LTXexample} 
$\\$
Notice the code \texttt{pbk}. You need to know this code\index{font codes} to access the font. The codes for the most common fonts can be found at \href{https://tex.stackexchange.com/questions/25249/how-do-i-use-a-particular-font-for-a-small-section-of-text-in-my-document/37251#37251}{Stack Exchange}. \\\\
Sometimes the font does not appear. This is because it needs to be installed by \LaTeX. For more information, read \href{https://tug.org/fonts/fontinstall.html}{this} guide.

% CHAPTER 4 % 
\chapter{International Language Support\index{languages}\index{other languages}\index{International Language Support}}

% Polyglossia Package % 
\section{Using \texttt{polyglossia}}

% Font Definition Languages % 
\subsection{Languages Requiring Font Definitions}
\label{languageswithfontdefinitions}
% Arabic %
\subsubsection{Arabic\index{Arabic}\index{International Language Support!Arabic}}
Use the \texttt{polyglossia} package with the \XeLaTeX{}\index{XeLaTeX@\XeLaTeX} compiler as follows:
\begin{lstlisting}[style=myLaTeX, columns=fullflexible]
% Preamble 
\usepackage{polyglossia}
\setdefaultlanguage{english}
\setotherlanguage{arabic}
\newfontfamily{\arabicfont}[Script=Arabic]{Scheherazade}
\end{lstlisting}
$\\$
\texttt{\textbackslash arabicfont} is used to type Arabic: {\arabicfont{يمكن عكس هذا النص}} \\\\ You must download the \href{https://www.1001fonts.com/scheherazade-font.html}{Scheherazade} font on your local PC. Other fonts that support Arabic can be found \href{https://www.overleaf.com/learn/latex/Questions\%2FWhich_OTF_or_TTF_fonts_are_supported_via_fontspec\%3F\#Fonts_for_Arabic_script}{here}.

% Bengali %
\subsubsection{Bengali\index{Bengali}\index{International Language Support!Bengali}}
Use the \texttt{polyglossia} package with the \XeLaTeX{}\index{XeLaTeX@\XeLaTeX} compiler as follows:
\begin{lstlisting}[style=myLaTeX, columns=fullflexible]
% Preamble 
\usepackage{polyglossia}
\setdefaultlanguage{english}
\setotherlanguage{bengali}
\newfontfamily{\bengalifont}[Script=Bengali]{Kalpurush}
\end{lstlisting}
$\\$
\texttt{\textbackslash bengalifont} is used to type Bengali: {\bengalifont{আমি বাংলা বলি না।}} \\\\ You must download the \href{http://banglafont.com/?p=Download\%20Kalpurush/}{Kalpurush} font on your local PC. Other fonts that support Bengali can be found \href{https://www.overleaf.com/learn/latex/Questions\%2FWhich_OTF_or_TTF_fonts_are_supported_via_fontspec\%3F\#Bengali}{here}.

% Greek %
\subsubsection{Greek\index{Greek}\index{International Language Support!Greek}}
Use the \texttt{polyglossia} package with the \XeLaTeX{}\index{XeLaTeX@\XeLaTeX} compiler as follows:
\begin{lstlisting}[style=myLaTeX, columns=fullflexible]
% Preamble 
\usepackage{polyglossia}
\setdefaultlanguage{english}
\setotherlanguage{greek}
\newfontfamily{\greekfont}[Script=Greek]{Linux Libertine}
\end{lstlisting}
$\\$
\texttt{\textbackslash greekfont} is used to type Greek: {\greekfont{Θέλω να πάω στη Μύκονο.}} \\\\ You must download the \href{https://www.dafont.com/linux-libertine.font}{Linux Libertine} font on your local PC. Some other Greek fonts can be found \href{https://www.overleaf.com/learn/latex/Questions\%2FWhich_OTF_or_TTF_fonts_are_supported_via_fontspec\%3F\#Fonts_for_Greek}{here}. 

% Hebrew %
\subsubsection{Hebrew\index{Hebrew}\index{International Language Support!Hebrew}}
Use the \texttt{polyglossia} package with the \XeLaTeX{}\index{XeLaTeX@\XeLaTeX} compiler as follows:
\begin{lstlisting}[style=myLaTeX, columns=fullflexible]
% Preamble %
\usepackage{polyglossia}
\setdefaultlanguage{english}
\setotherlanguage{hebrew}
\newfontfamily{\hebrewfont}[Script=Hebrew]{IBM Plex Sans Hebrew}
\end{lstlisting}
$\\$
\texttt{\textbackslash hebrewfont} is used to type Hebrew: {\hebrewfont{אנחנו אנשים אינטליגנטים.}} \\\\ You must download the \href{https://fonts.google.com/specimen/IBM+Plex+Sans+Hebrew?query=ibm}{IBM Plex Sans Hebrew} font on your local PC. Other fonts that support Hebrew can be found \href{https://www.overleaf.com/learn/latex/Questions\%2FWhich_OTF_or_TTF_fonts_are_supported_via_fontspec\%3F\#Fonts_for_Hebrew_script}{here}.

% Hindi %
\subsubsection{Hindi\index{Hindi}\index{International Language Support!Hindi}}
Use the \texttt{polyglossia} package with the \XeLaTeX{}\index{XeLaTeX@\XeLaTeX} compiler as follows:
\begin{lstlisting}[style=myLaTeX, columns=fullflexible]
% Preamble 
\usepackage{polyglossia}
\setdefaultlanguage{english}
\setotherlanguage{hindi}
\newfontfamily{\hindifont}[Script=Devanagari]{Lohit Devanagari}
\end{lstlisting}
$\\$
\texttt{\textbackslash hindifont} is used to type Hindi: {\hindifont{विराट कोहली भगवान हैं।}} \\\\ You must download the \href{https://www.ffonts.net/Lohit-Devanagari.font.download}{Lohit Devanagari} font on your local PC. Other fonts that support Hindi can be found \href{https://www.overleaf.com/learn/latex/Questions\%2FWhich_OTF_or_TTF_fonts_are_supported_via_fontspec\%3F\#Hindi_.28Devanagari_script.29}{here}.

% Thai %
\subsubsection{Thai\index{Thai}\index{International Language Support!Thai}}
Use the \texttt{polyglossia} package with the \XeLaTeX{}\index{XeLaTeX@\XeLaTeX} compiler as follows:
\begin{lstlisting}[style=myLaTeX, columns=fullflexible]
% Preamble 
\usepackage{polyglossia}
\setdefaultlanguage{english}
\setotherlanguage{thai}
\newfontfamily\thaifont[Script=Thai]{Prompt}
\end{lstlisting}
$\\$
\texttt{\textbackslash thaifont} is used to type Thai: {\thaifont{กุมารเป็นคนไทย}} \\\\ You must download the \href{https://fonts.google.com/specimen/Prompt?subset=thai}{Prompt} font on your local PC. Other fonts that support Thai can be found \href{https://www.overleaf.com/learn/latex/Questions\%2FWhich_OTF_or_TTF_fonts_are_supported_via_fontspec\%3F\#Fonts_for_Thai}{here}.

% Russian %
\subsubsection{Russian\index{Russian}\index{International Language Support!Russian}}
Use the \texttt{polyglossia} package with the \XeLaTeX{}\index{XeLaTeX@\XeLaTeX} compiler as follows:
\begin{lstlisting}[style=myLaTeX, columns=fullflexible]
% Preamble 
\usepackage{polyglossia}
\setdefaultlanguage{english}
\setotherlanguage{russian}
\newfontfamily\russianfont[Script=Cyrillic]{Linux Libertine}
\end{lstlisting}
$\\$
\texttt{\textbackslash russianfont} is used to type Russian: {\russianfont{Путин любит лошадей!}} \\\\ You must download the \href{https://www.dafont.com/linux-libertine.font}{Linux Libertine} font on your local PC. Some other Russian fonts can be found \href{https://www.lingualift.com/blog/cyrillic-russian-fonts/}{here}. \\\\ \textbf{If you set the default language as Russian, you can type it without \texttt{\textbackslash russianfont}. However, you still need to define \texttt{\textbackslash russianfont} using \texttt{\textbackslash newfontfamily} in the preamble. The same applies to the other languages in \ref{languageswithfontdefinitions}.}

% Other Languages % 
\subsection{Some Other Languages}

% Spanish %
\subsubsection{Spanish\index{Spanish}\index{International Language Support!Spanish}}
Use the \texttt{polyglossia} package with the \XeLaTeX{}\index{XeLaTeX@\XeLaTeX} compiler as follows:
\begin{LTXexample}[style=myLaTeX, pos=o]
% Preamble 
\usepackage{polyglossia}
\setdefaultlanguage{spanish} 

% Body
¡\today{} es mi cumpleaños!
\end{LTXexample} 
$\\$
If you only need to use Spanish temporarily:
\begin{LTXexample}[style=myLaTeX, pos=o]
% Preamble 
\usepackage{polyglossia}
\setdefaultlanguage{english} 
\setotherlanguage{spanish}

% Body
I speak Spanish. \\
\textspanish{\today{} es mi cumpleaños.}
\end{LTXexample} 

% French %
\subsubsection{French\index{French}\index{International Language Support!French}}
If you also want to add French, make the following changes:
\begin{LTXexample}[style=myLaTeX, pos=o]
% Preamble 
\usepackage{polyglossia}
\setdefaultlanguage{english} 
\setotherlanguages{spanish, french}

% Body
I speak Spanish. \\
\textspanish{\today{} es mi cumpleaños.} \\
\textfrench{Je parle aussi français.}
\end{LTXexample} 

% German %
\subsubsection{German\index{German}\index{International Language Support!German}}
You can simply replace \texttt{french} with \texttt{german} in the code above.

\paragraph{Read the \texttt{polyglossia} \href{https://ctan.org/pkg/polyglossia?lang=en}{documentation} for a full list of supported languages.}

% Languages Requiring Other Packages % 
\section{Languages Requiring Other Packages}

 % Chinese %
\subsubsection{Chinese\index{Chinese}\index{International Language Support!Chinese}}
The most comprehensive solution is to use the \texttt{ctex} package.
\begin{LTXexample}[style=myLaTeX, pos=o]
% Preamble 
\usepackage[UTF8]{ctex} 

% Body
你好
\end{LTXexample} 
$\\$
It is recommended to use the \XeLaTeX{}\index{XeLaTeX@\XeLaTeX} compiler.

% Japanese %
\subsubsection{Japanese\index{Japanese}\index{International Language Support!Japanese}}
The \texttt{xeCJK} package\footnote{This package can also be used to typeset Chinese.} takes care of Japanese.
\begin{LTXexample}[style=myLaTeX, pos=o]
% Preamble 
\usepackage{xeCJK}

% Body
日本の首都は東京です
\end{LTXexample} 
$\\$
You must use the \XeLaTeX{}\index{XeLaTeX@\XeLaTeX} compiler.   

\subsubsection{Helpful Resources\index{resources}\index{International Language Support!Resources}}
\begin{enumerate}
\item \href{https://www.overleaf.com/learn/latex/Questions/Which_OTF_or_TTF_fonts_are_supported_via_fontspec\%3F}{Language Fonts} - a list of fonts that support different languages.
\item \href{https://en.wikibooks.org/wiki/LaTeX/Internationalization}{Wikibooks} - a more comprehensive guide for typesetting different languages.
\item \href{https://www.overleaf.com/learn/latex/International_language_support\#Further_reading}{Overleaf} - further reading.
\end{enumerate}

% CHAPTER 6%
\chapter{Page Layout}

% Line Spacing %
\section{Line Spacing\index{line spacing}}
\command{linespread} alters the space between lines. Type it in the preamble.
\begin{lstlisting}[style=myLaTeX, columns=fullflexible]
% Default Line Spacing 
\linespread{1}

% One and a Half Line Spacing 
\linespread{1.3}

% Double Line Spacing 
\linespread{1.6}
\end{lstlisting} 
$\\$
\command{setlength}\texttt{\{\command{baselineskip}\}\{1.6 \command{baselineskip}\}} temporarily alters the line spacing to double spacing:
\begin{LTXexample}[style=myLaTeX, pos=o]
{
\setlength{\baselineskip}{1.6 \baselineskip}
A double spaced paragraph is a paragraph with twice the space between lines. Wow, that's meta!
\par} $\\$

This is a normal paragraph with normal spacing. Nothing special.
\end{LTXexample} 
$\\$
\command{par} ends a paragraph and is necessary\footnote{ You can also use an empty line.}.

% Paragraphs %
\section{Paragraphs\index{paragraphs}}

% Indentation %
\subsection{Paragraph Indentation\index{paragraph indentation}\index{indentation}}

% Temp Indentation %
\subsubsection{Temporary Indentation\index{temporary indentation}}
Sometimes paragraphs are indented. \command{noindent} cancels the indent. If you want to indent a non-indented paragraph, use \command{indent}.  

% Permanent Indentation %
\subsubsection{Permanent Indentation\index{permanent indentation}}
\command{setlength}\texttt{\{\command{parindent}\}\{4em\}} globally sets paragraph indentation to 4em. If you don't want any indentation in the document, use \command{setlength}\texttt{\{\command{parindent}\}\{0em\}}. Place the commands in the preamble, preferably before \command{tableofcontents}.

% Paragraph Spacing %
\subsection{Paragraph Spacing\index{paragraph spacing}\index{spacing}}
\command{setlength}\texttt{\{\command{parskip}\}\{1em\}} globally sets spacing between 2 paragraphs to 1em.\\\\
More information of units such as em can be found \href{https://tex.stackexchange.com/questions/8260/what-are-the-various-units-ex-em-in-pt-bp-dd-pc-expressed-in-mm}{here}.

% Paragraph Shape %
\subsection{Paragraph Shape\index{paragraph shape}\index{shaping paragraphs}}
Load the \texttt{shapepar} package. You can write paragraphs with cool shapes.
\begin{LTXexample}[style=myLaTeX, pos=o]
\heartpar{This paragraph is shaped as a heart because \LaTeX{} is powerful. There are many other shapes available. Just go through the documentation for \texttt{shapepar}. There are circles, squares, rectangles, and shapes you can't imagine. This is kind of cringe but oh well.}
\end{LTXexample} 
$\\$
For more information, read the \texttt{shapepar} \href{https://ctan.org/pkg/shapepar}{documentation}. For irregular shapes, read this \href{https://tex.stackexchange.com/questions/32226/how-to-layout-irregular-paragraph-shape}{post}. \\\\
More information on paragraph formatting can be found on \href{https://www.overleaf.com/learn/latex/Paragraph_formatting}{Overleaf}.

% Page Elements % 
\section{Page Elements \index{page elements}}
All documents classes except \texttt{book} (refer to \ref{classes} on page \pageref{classes}) are one-sided. In one-sided documents, each page is identical.  In two-sided documents, odd and even pages have different margins. To create a two-sided document, use the \texttt{twoside} option.
\begin{lstlisting}[style=myLaTeX, columns=fullflexible]
% Books are two-sided documents
\documentclass{book}

% Use the two-sided option for other document classes
\documentclass[twoside]{article}
\end{lstlisting} 
$\\$
More information on one/two-sided documents can be found \href{https://www.overleaf.com/learn/latex/Single_sided_and_double_sided_documents}{here}.

% Headers & Footers % 
\subsection{Headers \& Footers \index{header}\index{footer}}

% Basic Customization %
\subsubsection{Basic Customization}
Use \command{pagestyles} in the preamble.
\begin{lstlisting}[style=myLaTeX, columns=fullflexible]
% No Header, No Footer
\pagestyles{empty} 
\end{lstlisting} 
\begin{lstlisting}[style=myLaTeX, columns=fullflexible]
% No Footer, Header contains page number and some information
\pagestyles{headings} 
\end{lstlisting} 
\begin{lstlisting}[style=myLaTeX, columns=fullflexible]
% No Footer, Header for one-sided document
\pagestyles{myheadings} 
\markright{Name \hfill Date \hfill} % Name is placed on the left, Date is placed in the center, page number on the right
\end{lstlisting} 

\begin{lstlisting}[style=myLaTeX, columns=fullflexible]
% No Footer, Header for two-sided document
\pagestyles{myheadings} 
\markboth{Hi}{Hello} % 'Hi' on even pages, 'Hello' on odd pages
\end{lstlisting} 
\indexcommand{markright}\indexcommand{markboth}

% Advanced Customization %
\subsubsection{Advanced Customization}
Load the \texttt{fancyhdr} package and do the following:
\begin{lstlisting}[style=myLaTeX, columns=fullflexible]
% Preamble
\usepackage{fancyhdr}
\pagestyle{fancy} % Sets page style to fancy
\fancyhf{} % Clears the Header and Footer for customization
\end{lstlisting} 
$\\$
For a one-sided documents, use the following commands:
\begin{lstlisting}[style=myLaTeX, columns=fullflexible]
% Preamble
\rhead{Right Side of Header}
\chead{Header Center}
\lchead{Left Side of Header}
\rfoot{Right Side of Footer}
\cfoot{Footer Center} 
\lfoot{Left Side of Footer}
\end{lstlisting} 
$\\$
For two-sided documents:
\begin{lstlisting}[style=myLaTeX, columns=fullflexible]
% Preamble
\fancyhead[LE,RO]{Outer}  % LE = Left Even; RO = Right Odd
\fancyhead[RE,LO]{Inner} % RE = Right Even; LO = Left Odd
\fancyfoot[CE,CO]{Center} % CE = Center Even; CO = Center Odd
\fancyfoot[LE,RO]{\thepage} % Prints the page number on the Footer center on even and odd pages
\end{lstlisting} 
$\\$
More commands include:
\begin{table}[H]
\centering
\small
\begin{tabular}{|cc|}
\hline
\textbf{Description}                             & \textbf{Command}      \\                               
Page Number & \command{thepage} \\
Chapter Number & \command{thechapter}  \\ 
Section Number & \command{thesection} \\ 
Chapter Name & \command{chaptername}  \\ 
Current Chapter / Section Name \& Number & \command{leftmark}  \\ 
Current Section / Subsection Name \& Number & \command{rightmark}  \\ 
\hline
\end{tabular}
\end{table}
\index{chapter name}\index{chapter number}
\index{section name}\index{section number}
\index{page number}
\noindent\texttt{fancyhdr} provides decorative lines for the header and footer. If you need to customize the lines, do the following: 
\begin{lstlisting}[style=myLaTeX, columns=fullflexible]
% Preamble
\renewcommand{\headrulewidth}{2pt} % Header Line
\renewcommand{\footrulewidth}{2pt} % Footer Line
\end{lstlisting} 

% Page Numbers % 
\subsection{Page Numbers\index{page numbers}}
Use \command{pagenumbering}.
\begin{lstlisting}[style=myLaTeX, columns=fullflexible]
% Preamble
\pagenumbering{arabic}
\end{lstlisting} 
$\\$
The page numbers will be Arabic numerals. If you need lowercase (uppercase) Roman numerals, use \texttt{roman} (\texttt{Roman}) instead. \\\\ \command{setcounter}\texttt{\{page\}} allows you to control the page counter.
\begin{lstlisting}[style=myLaTeX, columns=fullflexible]
\chapter{First Chapter}
\setcounter{page}{3} % 3 is assigned to the current page
\end{lstlisting} 
$\\$
More information on page numbering can be found \href{https://www.overleaf.com/learn/latex/Page_numbering}{here}.

% Margins %
\section{Page Margins \index{page size}\index{page margin}}
The \texttt{geometry} package allows you to change the margins. 
\begin{lstlisting}[style=myLaTeX, columns=fullflexible]
% Preamble
\usepackage[margin=2in]{geometrty}
\end{lstlisting} 
$\\$
I have also used it to create help sheets.
\begin{lstlisting}[style=myLaTeX, columns=fullflexible]
% Preamble
\documentclass{article}
\usepackage[margin=1cm, landscape]{geometry}
\usepackage{multicol}

% Body
\begin{multicols}{3}
\section{Section 1}

\section{Section 2}

\section{Section 3}

\section{Section 4}
\end{multicols}
\end{lstlisting} 

% Page Layout LaTeX Resources %
\subsubsection{Helpful Resources\index{resources}}
Frankly, I don't have much experience with page margins, so I direct you to the following resources.
\begin{enumerate}
\item \href{https://ctan.org/pkg/geometry?lang=en}{CTAN} - \texttt{geometry} package documentation.
\item \href{https://www.overleaf.com/learn/latex/Page_size_and_margins}{Overleaf} - an introduction to page size and margins. 
\item \href{https://en.wikibooks.org/wiki/LaTeX/Page_Layout}{Wikibooks} - an advanced guide for page layout (page margins, page size, page background).
\end{enumerate}
\index{page background}\index{background}

% CHAPTER 6%
\chapter{Mathematics}
This is where the heart of \LaTeX{} lies.

% AMS Package %
\section{\AmS-\LaTeX{} packages\index{American Mathematical Society}}
It is \textbf{strongly recommended} to load the packages in the \AmS-\LaTeX{} bundle when typesetting math. Copy this into the document's preamble:
\begin{lstlisting}[style=myLaTeX, columns=fullflexible]
\usepackage{mathtools, amssymb, amsthm}
\end{lstlisting} 

% Math Mode %
\section{Math Mode\index{math mode}}
\label{math mode}
Math mode must be used when writing math. There are 2 modes: inline and display.

% Inline %
\subsection{Inline (Text)\index{inline}\index{text mode}}
Inserts math in between text. Requires:
\begin{itemize}
\item an opening \texttt{\$}\index{dollar sign}\indexcommand{\$} and closing \$;
\item or an opening \texttt{\textbackslash(}and closing \texttt{\textbackslash)}; 
\item or a \texttt{math} environment\index{environments!math@\texttt{math}}.
\end{itemize}
\begin{LTXexample}[style=myLaTeX, pos=o]
$9+10=21$ is false. \\

\(3^2 + 4^2 = 5^2\) is true. \\

\begin{math}
\sin{x}=\pi \text{ has no solutions.}
\end{math}

\end{LTXexample} 
$\\$ 
\command{text} inserts text in math mode. If not, this happens: 
\begin{LTXexample}[style=myLaTeX, pos=o]
\begin{math}
\sin{x} = \pi has no solutions.
\end{math}
\end{LTXexample} 
\index{text in math mode}

% Display (unnumbered) %
\subsection{Display (unnumbered)\index{equation}\index{display mode}\index{unlabelled equation}\index{unnumbered equation}}
Displays math in its own line and is unnumbered. Requires:
\begin{itemize}
\item an opening \texttt{\textbackslash[} and closing \texttt{\textbackslash]};
\item or an \texttt{equation*} environment\index{environments!equation*@\texttt{equation*}};
\item or a \texttt{displaymath} environment\index{environments!display@\texttt{displaymath}}.
\end{itemize}
\begin{LTXexample}[style=myLaTeX, pos=o]
The Pythagorean theorem: \[a^2+b^2=c^2\]

Environments also work:
\begin{equation*}
a^2+b^2=c^2
\end{equation*}

Another environment:
\begin{displaymath}
a^2+b^2=c^2
\end{displaymath}
\end{LTXexample} 
$\\$
Using an opening \verb|$$| and closing \verb|$$| also works but is \textbf{strongly discouraged}. Read \href{https://tex.stackexchange.com/questions/503/why-is-preferable-to}{this} for more information. 

% Display (numbered) %
\subsection{Display (numbered)\index{equation}\index{display mode}\index{labelled equation}\index{numbered equation}}
Displays math in its own line and is automatically labelled. Requires:
\begin{itemize}
\item an \texttt{equation} environment\index{environments!equation@\texttt{equation}}.
\end{itemize}
\begin{LTXexample}[style=myLaTeX, pos=o]
A legend once said that
\begin{equation} 
9+10=21
\end{equation}
\end{LTXexample} 

% Math Mode Fonts %
\subsection{Math Mode Fonts\index{math fonts}\index{fonts}}
The text and math fonts in a \LaTeX{} document are independent. The default math font is \textrm{Computer Modern}. Math fonts compatible with \LaTeX{} are rare. Refer to the footnote in \ref{mathfont} for more information. \\\\ If you need to use math fonts temporarily, \LaTeX{} provides a few pre-defined commands:
\begin{table}[H]
\centering
\small
\begin{tabular}{|ccc|}
\hline
\textbf{Description}                             & \textbf{Command}                           & \textbf{Output}    \\                              
Default & ABCabc123 & $ABCabc123$ \\ 
Roman & \command{mathrm}\texttt{\{ABCabc123\}} & $\mathrm{ABCabc123}$ \\ 
Bold & \command{mathbf}\texttt{\{ABCabc123\}} & $\mathbf{ABCabc123}$ \\ 
Italics & \command{mathit}\texttt{\{ABCabc123\}} & $\mathit{ABCabc123}$ \\ 
Typewriter & \command{mathtt}\texttt{\{ABCabc123\}} & $\mathtt{ABCabc123}$ \\ 
Fraktur & \command{mathfrak}\texttt{\{ABCabc123\}} & $\mathfrak{ABCabc123}$ \\ 
Blackboard Bold & \command{mathbb}\texttt{\{ABC\}} & $\mathbb{ABC}$ \\ 
Caligraphic & \command{mathcal}\texttt{\{ABC\}} & $\mathcal{ABC}$ \\ 
\hline
\end{tabular}
\end{table}
\index{bold}
\index{italics}
\index{typewriter}\index{monospace}
\index{blackboard bold}
\index{caligraphic letters}
\index{fraktur}
\noindent Note that \command{mathbb} and \command{mathcal} only work with capital letters.

% Math Mode Spacing %
\subsection{Math Mode Spacing\index{math spacing}\index{spacing}}
\LaTeX{} automatically spaces content in math mode. It ignores whitespace characters. If you want custom spacing, refer to this \href{https://www.overleaf.com/learn/latex/Spacing_in_math_mode}{table}. Here are a few examples:
\begin{LTXexample}[style=myLaTeX, pos=o]
$1     2 $ \\
$1 \! 2 $ \\
$1 \, 2 $ \\
$1 \: 2 $ \\
$1 \; 2 $ \\
$1 \  2 $ \\
$1 \quad 2 $ \\
$1 \qquad 2 $ \\
$1 \qquad \quad 2 $ \\
$1 \qquad \qquad 2 $ \\
$1 \hspace*{4cm} 2 $
\end{LTXexample} 
$\\$
\command{phantom} acts like a whitespace character. 
\begin{LTXexample}[style=myLaTeX, pos=o]
${}^{14}_{6}\mathrm{C}$ \\ 
${}^{14}_{\phantom{1}6}\mathrm{C}$  % 1 space reserved
\end{LTXexample} 

% Equations %
\section{Equations\index{equations}}
\label{equations}
% Labelling Equations %
\subsection{Labelling Equations \index{labelling equations}}
\LaTeX{} automatically labels equations. Use \command{tag} for a custom label.
\begin{LTXexample}[style=myLaTeX, pos=o]
A legend once said that
\begin{equation} 
9+10=21 \tag{genius}
\end{equation}
\end{LTXexample}
$\\$
\command{label} and \command{eqref} allow you to refer to an equation.
\begin{LTXexample}[style=myLaTeX, pos=o]
\begin{equation} 
1+1=2  \label{babymath}
\end{equation}
This refers to \eqref{babymath}.
\end{LTXexample}

% Long Equations %
\subsection{Long Equations \index{long equations}}
Sometimes equations are too long.
\begin{LTXexample}[style=myLaTeX, pos=o]
A long sum:
\[91=1+2+3+4+5+6+7+8+9+10+11+12+13\]
\end{LTXexample}
$\\$
The \texttt{multline} environment resolves this. \index{environments!multline@\texttt{multline}}
\begin{LTXexample}[style=myLaTeX, pos=o]
A long sum:
\begin{multline}
91=1+2+3+4+5 \\
+6+7+8+9+10 \\
+11+12+13
\end{multline}
\end{LTXexample}
$\\$
\texttt{multline*} removes the equation label.

% Gathering Equations %
\subsection{Gathering Equations \index{gather equations}}
The \texttt{multline} solution looks messy, so another solution is to gather equations. The \texttt{gather} environment brings equations together, centered. \index{environments!gather@\texttt{gather}}
\begin{LTXexample}[style=myLaTeX, pos=o]
Centered equations:
\begin{gather}
1+1=2 \\
xyz+x+y+z=w
\end{gather}
\end{LTXexample}

\begin{LTXexample}[style=myLaTeX, pos=o]
Centered (unnumbered) equations:
\begin{gather*}
1+1=2 \\
xyz+x+y+z=w
\end{gather*}
\end{LTXexample}
$\\$
The \texttt{gathered} environment (\textbf{within display mode}) assigns 1 label to the gathered equations.\index{environments!gathered@\texttt{gathered}}
\begin{LTXexample}[style=myLaTeX, pos=o]
\begin{equation}
\begin{gathered}
1+1=2 \\
xyz+x+y+z=w
\end{gathered}
\end{equation}
\end{LTXexample}

% Aligning Equations %
\subsection{Aligning Equations \index{align equations}}
Another solution to long equations is to align equations on the relation symbol ($=$, $<$, etc.) using the \texttt{align} environment.\index{environments!align@\texttt{align}}
\begin{LTXexample}[style=myLaTeX, pos=o]
Aligned equations:
\begin{align}
1+1=2 \\
xyz+x+y+z=w
\end{align}
\end{LTXexample}

\begin{LTXexample}[style=myLaTeX, pos=o]
Aligned (unnumbered) simplification:
\begin{align*}
1+2+3+4&=1+2+7 \\
&=1+9 \\
&=10
\end{align*}
\end{LTXexample}
$\\$
Add text using \texttt{\&\&} and \command{text}.
\begin{LTXexample}[style=myLaTeX, pos=o]
Simplifying $1+2+3+4$:
\begin{align*}
1+2+3+4&=1+2+7 \\
&=1+9 && \text{(as $2+7=9$)} \\
&= 10
\end{align*}
\end{LTXexample}
$\\$
Some more equations:
\begin{LTXexample}[style=myLaTeX, pos=o]
Aligned (unnumbered) equations:
\begin{align*}
x&=y & y&=z  \\
z&=x & y&=x  \\
z&=y & x&=z 
\end{align*}
\end{LTXexample}
$\\$
The \texttt{aligned} environment (\textbf{within display mode}) assigns 1 label to the aligned equations.\index{environments!aligned@\texttt{aligned}}
\begin{LTXexample}[style=myLaTeX, pos=o]
\begin{equation}
\begin{aligned}
1+1=2 \\
xyz+x+y+z=w
\end{aligned}
\end{equation}
\end{LTXexample}
$\\$
You can also use \texttt{split} instead of \texttt{aligned}.\index{environments!split@\texttt{split}} Read more about the differences \href{https://tex.stackexchange.com/questions/187938/whats-the-difference-between-split-and-aligned}{here}.

% Math Environments %
\section{Math Environments\index{math environments}} 

% Statement Environments %
\subsection{Theorems, Lemmas, Corollaries, Propositions, Definitions, Remarks, Examples, Exercises, Asides}
\label{mathenvironments}
These environments may be helpful when writing math. Define them in the preamble.
\begin{lstlisting}[style=myLaTeX, columns=fullflexible]
\newtheoremstyle{dotless}{}{}{}{}{\bfseries}{}{12pt}{}
\theoremstyle{dotless}
\newtheorem{thm}{Theorem}[section] % Theorem 
\newtheorem*{thm*}{Theorem} % Theorem (unnumbered)
\newtheorem{lem}[thm]{Lemma} % Lemma 
\newtheorem*{lem*}{Lemma} % Lemma (unnumbered)
\newtheorem{cor}[thm]{Corollary} % Corollary 
\newtheorem*{cor*}{Corollary} % Corollary (unnumbered)
\newtheorem{prop}[thm]{Proposition} % Proposition
\newtheorem*{prop*}{Proposition} % Proposition (unnumbered)
\newtheorem{defn}[thm]{Definition} % Definition
\newtheorem*{defn*}{Definition} % Definition (unnumbered)
\newtheorem{rem}[thm]{Remark} % Remark
\newtheorem*{rem*}{Remark} % Remark (unnumbered)
\newtheorem{exa}[thm]{Example} % Example
\newtheorem*{exa*}{Example} % Example (unnumbered)
\newtheorem{exe}[thm]{Exercise} % Exercise
\newtheorem*{exe*}{Exercise} % Exercise (unnumbered)
\newtheorem{aside}[thm]{Aside} % Aside
\newtheorem*{aside*}{Aside} % Aside (unnumbered)
\end{lstlisting} 
\index{theorem}
\index{lemma}
\index{corollary}
\index{proposition}
\index{definition}
\index{remark}
\index{example}
\index{exercise}
\index{aside}
$\\$
Implementing:
\begin{LTXexample}[style=myLaTeX, pos=o]
% Theorem
\begin{thm}
This is a theorem.
\end{thm}

% Example
\begin{exa}
$1+1=2$
\end{exa}

% Exercise (unnumbered)
\begin{exe*}
Does $2+2=4$?
\end{exe*}
\end{LTXexample}

% Proofs %
\subsection{Proofs \index{proofs}}
\label{proofs}
The \texttt{proof} environment\index{environments!proof@\texttt{proof}} is provided by the \texttt{mathtools} package.
\begin{LTXexample}[style=myLaTeX, pos=o]
\begin{proof}
$1+1=2 \implies 1=1$
\end{proof}
\end{LTXexample}
$\\$
Changing the QED symbol:
\begin{LTXexample}[style=myLaTeX, pos=o]
% Preamble
\renewcommand\qedsymbol{$\blacksquare$}

% Body
\begin{proof}
$1+1=2 \implies 1=1$
\end{proof}
\end{LTXexample}
$\\$
Changing the QED symbol (again):
\begin{LTXexample}[style=myLaTeX, pos=o]
% Preamble
\renewcommand\qedsymbol{QED}

% Body
\begin{proof}
$1+1=2 \implies 1 = 1$
\end{proof}
\end{LTXexample}
$\\$ 
You can also change the style of the \texttt{proof} environment.
\begin{LTXexample}[style=myLaTeX, pos=o]
% Preamble
\renewenvironment{proof}{{\bfseries Proof.}}{\hfill$\square$}

% Body
\begin{proof}
$1+1=2 \implies 1 = 1$
\end{proof}
\end{LTXexample}
$\\$
Unlike the environments in \ref{equations}, math mode must be used in the environments mentioned in \ref{mathenvironments} and \ref{proofs}.

% Numbers %
\section{Numbers\index{numbers}} 

% Reals %
\subsection{Reals ($\mathbb{R}$)\index{reals}\index{numbers}}

% Integers %
\subsubsection{Integers ($\mathbb{Z}$)\index{integers}\index{whole numbers}}
Type out the integers. The font changes in math mode.
\begin{LTXexample}[style=myLaTeX, pos=o]
$-1, 0, 2, 4$ \\ % math mode font
-1, 0, 2, 4 % text mode font 
\end{LTXexample} 

% Rationals %
\subsubsection{Rationals ($\mathbb{Q}$)\index{rationals}\index{fractions}}
\command{frac} is used.
\begin{LTXexample}[style=myLaTeX, pos=o]
$\frac{1}{2}$ \\

$\frac{\frac{1}{2}+\frac{1}{2}}{1+2}$

\[\frac{7}{10}\] 
\end{LTXexample} 
$\\$
Use \command{dfrac} for a \textbf{d}isplay mode sized fraction and \command{tfrac} for a \textbf{t}ext mode sized fraction.
\begin{LTXexample}[style=myLaTeX, pos=o]
$\dfrac{69}{420}$ 
\[\tfrac{1}{1000}\] 
\end{LTXexample} 
$\\$
You can also use \texttt{\^{}\{1\}/\_{}\{2\}} to display $^{1}/_{2}$.

% Irrationals %
\subsubsection{Irrationals($\mathbb{I}$)\index{irrationals}\index{Pi}\index{Euler's Number}\index{logarithms}\index{Golden Ratio}\index{square roots}}
A few famous irrationals: 
\begin{table}[H]
\centering
\small
\begin{tabular}{|ccc|}
\hline
\textbf{Description}                             & \textbf{Command}                           & \textbf{Output}    \\                              
Pi    & \command{pi}     & $\pi$ \\ 
Euler's Number    & \texttt{e}, \texttt{\textbackslash mathrm\{e\}}  & $e$, $\mathrm{e}$ \\ 
Logarithms  & \command{log}\texttt{\_\{2\}\{3\}}    & $\log_{2}{3}$ \\ 
Golden Ratio   & \command{phi}     & $\phi$ \\ 
Unit Square Diagonal & \command{sqrt}\texttt{\{2\}} & $\sqrt{2}$ \\
\hline
\end{tabular}
\end{table}
\begin{LTXexample}[style=myLaTeX, pos=o]
The area of a unit circle is $\pi$.
\end{LTXexample}

% Complex Numbers %
\subsection{Complex Numbers ($\mathbb{C}$)\index{complex numbers}\index{imaginary numbers}}
\begin{LTXexample}[style=myLaTeX, pos=o]
Imaginary Unit: $i$, $\mathrm{i}$
\end{LTXexample}
\begin{LTXexample}[style=myLaTeX, pos=o]
Complex number: $z=2+2i$
\end{LTXexample}
\begin{LTXexample}[style=myLaTeX, pos=o]
Conjugate: $z^{*}=2-2i$
\end{LTXexample}
\begin{LTXexample}[style=myLaTeX, pos=o]
Real Part: $\Re(z)=2$ or $\mathrm{Re}(z)=2$
\end{LTXexample}
\begin{LTXexample}[style=myLaTeX, pos=o]
Imaginary Part: $\Im(z)=2$ or $\mathrm{Im}(z)=2$
\end{LTXexample}
\begin{LTXexample}[style=myLaTeX, pos=o]
Absolute Value: $|z|=2\sqrt{2}$
\end{LTXexample}
\begin{LTXexample}[style=myLaTeX, pos=o]
Argument: $\arg(z)=\frac{\pi}{2}$
\end{LTXexample}

% Variables %
\section{Variables\index{variables}\index{algebra}}
Variables are letters in math mode.
\begin{LTXexample}[style=myLaTeX, pos=o]
Find x vs. Find $x$
\end{LTXexample}
\begin{LTXexample}[style=myLaTeX, pos=o]
% Text Mode
abcdefghijklmnopqrstuvwxyz

% Math Mode
$abcdefghijklmnopqrstuvwxyz$ 
\end{LTXexample}
\begin{LTXexample}[style=myLaTeX, pos=o]
The solution to $ax^2+bx+c=0$ is \[x=\frac{-b \pm \sqrt{b^2-4ac}}{2a}.\]
\end{LTXexample}

% Greek Letters %
\subsection{Greek \& Hebrew Letters\index{Greek Letters}\index{Hebrew Letters}\index{other letters}}
\begin{table}[H]
\centering
\small
\begin{tabular}{|cc|cc|cc|cc|}
\hline
\textbf{Command}                             & \textbf{Output}                           & \textbf{Command}                                     & \textbf{Output}    & \textbf{Command}                                     & \textbf{Output}  & \textbf{Command}                                     & \textbf{Output}          \\
\command{alpha}      & $\alpha$      & \command{mu} & $\mu$ & \command{upsilon} & $\upsilon$ & \command{Upsilon} & $\Upsilon$      \\
\command{beta}           & $\beta$       & \command{nu} & $\nu$  & \command{xi} & $\xi$ & \command{aleph} & $\aleph$       \\
\command{chi}       & $\chi$       & \command{omega}  & $\omega$ & \command{zeta} & $\zeta$ & \command{beth} & $\beth$     \\
\command{delta}       & $\delta$      & \command{phi} & $\phi$  & \command{Delta} & $\Delta$ & \command{daleth} & $\daleth$       \\
\command{epsilon}     & $\epsilon$    & \command{varphi}  &  $\varphi$ & \command{Gamma} & $\Gamma$ & \command{gimel} & $\gimel$      \\
\command{varepsilon}   & $\varepsilon$ & \command{pi}   &  $\pi$  & \command{Lambda} & $\Lambda$ & &   \\
\command{eta} & $\eta$        &   \command{psi}   & $\psi$ & \command{Omega} & $\Omega$  & &    \\
\command{gamma}    & $\gamma$      &     \command{rho}                               & $\rho$ & \command{Phi} & $\Phi$    & &   \\
\command{iota}     & $\iota$     &         \command{sigma}                           &   $\sigma$  & \command{Pi} & $\Psi$ & & \\
\command{kappa}    & $\kappa$    &            \command{tau}                        & $\tau$  & \command{Sigma} & $\Sigma$   & &  \\
\command{lambda}& $\lambda$     &                    \command{theta}                & $\theta$ & \command{Theta} & $\Theta$& &  \\
\hline
\end{tabular}
\end{table}
\begin{LTXexample}[style=myLaTeX, pos=o]
The cardinality of the natural numbers is $\aleph_0$.
\end{LTXexample}
$\\$
More Greek symbols can be found \href{https://oeis.org/wiki/List_of_LaTeX_mathematical_symbols\#Greek_letters}{here}.

% Math Mode Symbols %
\section{Math Symbols\index{symbols}\index{math symbols}}

In this section, I will list some basic math symbols. However, if you need to find commands for symbols that are not included here, refer to \ref{symbols} on page \pageref{symbols}.

% Basic Arithmetic Symbols %
\subsection{Basic Arithmetic\index{arithmetic}}
\begin{table}[H]
\centering
\small
\begin{tabular}{|ccc|}
\hline
\textbf{Description}                             & \textbf{Command}                           & \textbf{Output}    \\                              
Addition & \texttt{+} & $+$ \\ 
Subtraction & \texttt{-} & $-$ \\ 
Multiplication (times) & \command{times} & $\times$ \\ 
Multiplication (dot) & \command{cdot} & $\cdot$ \\ 
Division (sign) & \command{div} & $\div$ \\ 
Division (slash) & \texttt{/} & $/$ \\ 
Exponentiation & \texttt{a\^{}\{b\}} & $a^b$ \\ 
\hline
\end{tabular}
\end{table}
\index{addition}
\index{subtraction}\index{minus}
\index{multiplication}\index{product}\index{times}
\index{division}
\index{exponentiation}\index{power}\index{superscripts}
\begin{LTXexample}[style=myLaTeX, pos=o]
$3 \times ((3^{3} \div 3)+(((3+3)/3)+3-3+3+3+3+3))=69$
\end{LTXexample}

% Basic Algebra %
\subsection{Basic Algebra\index{algebra}}
\begin{table}[H]
\centering
\small
\begin{tabular}{|ccc|}
\hline
\textbf{Description}                             & \textbf{Command}                           & \textbf{Output}    \\ 
Plus-Minus & \command{pm}& $\pm$ \\ 
Minus-Plus & \command{mp}& $\mp$ \\                                       
Square Root & \command{sqrt}\texttt{\{x\}} & $\sqrt{x}$ \\ 
$n^{\text{th}}$ Root & \command{sqrt}\texttt{[n]\{x\}} & $\sqrt[n]{x}$ \\ 
Absolute Value & \texttt{|x|} & $|x|$ \\ 
Natural Log & \command{ln}\texttt{\{x\}},  \command{ln}\texttt{(x+y)}& $\ln{x}$, $\ln(x+y)$ \\ 
Log & \command{log}\texttt{\_{}\{2\}\{3\}}, \command{log}\texttt{\_{}\{2\}(3+1)} & $\log_{2}(3+1)$ \\
Factorial & \texttt{5!}& $5!$ \\              
\hline
\end{tabular}
\end{table}
\index{plus-minus}
\index{minus-plus}
\index{factorial}
\index{square root}
\index{nth root}\index{cube root}
\index{absolute value}\index{modulus}
\index{natural log}\index{log}\index{ln}
\index{log}
\begin{LTXexample}[style=myLaTeX, pos=o]
$\sqrt{\sqrt{81}}=3$ \\
$\sqrt[2]{9}=3$ \\
$\sqrt[3]{27}=3$ \\
$|3|=3$ \\
$\ln{3}=\log_{e}{3} $ \\
$\ln(3)=\log_{e}(3)$
\end{LTXexample}
$\\$
More operator symbols can be found \href{https://oeis.org/wiki/List_of_LaTeX_mathematical_symbols\#Binary_operators}{here}.

% Polynom Package %
\subsubsection{\texttt{polynom} package}
You can divide and factorize polynomials with the \texttt{polynom} package.
\begin{LTXexample}[style=myLaTeX, pos=o]
% Long Division
$\polylongdiv{x^3+x^2+x+1}{x+1}$
\end{LTXexample}
\begin{LTXexample}[style=myLaTeX, pos=o]
% Factorization
$x^5+x^4+x^3+x^2+x+1 = $
\[\polyfactorize{
x^5+x^4+x^3+x^2+x+1
}\]
\end{LTXexample}

% Cancel Package %
\subsubsection{\texttt{cancel} package}
You can cancel fractions with the \texttt{cancel} package.
\begin{LTXexample}[style=myLaTeX, pos=o]
\[\frac{3 \cdot \cancel{2}}{\cancel{2} \cdot 4} = \frac{3}{4}\]
\end{LTXexample}


% Comparison Symbols %
\subsection{Comparison Symbols\index{comparison symbols}}
\begin{table}[H]
\centering
\small
\begin{tabular}{|ccc|}
\hline
\textbf{Description}                             & \textbf{Command}                           & \textbf{Output}    \\                              
Equal to& \texttt{=} & $=$ \\ 
Approximately Equal to& \command{approx} & $\approx$ \\ 
Not Equal to & \command{neq} & $\neq$ \\ 
Less Than & \texttt{<} & $<$ \\ 
Less Than or Equal to & \command{leq} & $\leq$ \\ 
Less Than or Equal to (slant) & \command{leqslant} & $\leqslant$ \\ 
Much Less Than & \command{ll} & $\ll$ \\
Greater Than & \texttt{>} & $>$ \\ 
Greater Than or Equal to & \command{geq} & $\geq$ \\ 
Greater Than or Equal to (slant) & \command{geqslant} & $\geqslant$ \\ 
Much Greater Than & \command{gg} & $\gg$ \\
Proportional To & \command{propto} & $\propto$ \\
\hline
\end{tabular}
\end{table}
\index{equal}
\index{approximately equal}
\index{not equal}
\index{less than}
\index{greater than}
\begin{LTXexample}[style=myLaTeX, pos=o]
$\pi \approx 3.14$, but $\pi \neq 3.14$ and $\pi \geq 3.14$. In other words, $\pi > 3.14$.
\end{LTXexample}
$\\$
More Comparison symbols can be found \href{https://oeis.org/wiki/List_of_LaTeX_mathematical_symbols\#Relation_operators}{here}.

% Superscripts & Subscripts %
\subsection{Su$_\text{b}^\text{per}$scripts\index{superscripts}\index{exponents}\index{subscripts}}
\begin{table}[H]
\centering
\small
\begin{tabular}{|ccc|}
\hline
\textbf{Description}                             & \textbf{Command}                           & \textbf{Output}    \\                              
Superscript & \texttt{a\^{}\{b\}} & $a^b$ \\ 
Subscript & \texttt{a\_{}\{0\}} & $a_0$ \\ 
\hline
\end{tabular}
\end{table}
\noindent \texttt{\^{}} and \texttt{\_{}} only impact the next character, so it's better to group the necessary characters with \texttt{\{\}}. \\\\
Examples with \texttt{\{\}}\footnote{\texttt{x\^{}2\^{}3} and \texttt{x\_{}2\_{}3} yield errors.}:
\begin{LTXexample}[style=myLaTeX, pos=o]
$x^{2} \quad x_{2}$ \\
$x^{10} \quad x_{10}$ \\
$x^{y^{2}} \quad x_{y_{2}}$ \\
$x^{y_{2}} \quad x_{y^{2}}$
\end{LTXexample}
$\\$
Examples without \texttt{\{\}}:
\begin{LTXexample}[style=myLaTeX, pos=o]
$x^2 \quad x_2$ \\
$x^2z^3 \quad x_2z_3$ \\
$x  ^2 z^  3 \quad x  _2 z_   2$ \\
$x^10 \quad x _10$ 
\end{LTXexample}
$\\$
Mixing things up:
\begin{LTXexample}[style=myLaTeX, pos=o]
$x^{420}_{69} - y \quad x_{69}^{420} - y$ \\
${}^{14}_{6}C \quad {}_{6}^{14}C$ \\
${}^1_1H \quad {}^1{}_1H$ \\
$A_1^2 \quad A{}_1{}^2$ \\
$A{}_1^2 \quad A_1{}^2$ \\
${}^{2}B^{2} \quad {}_2A_{1}$ \\
$x_1' \quad x''_2$ 
\end{LTXexample}


% Fences %
\subsection{Fences (Delimiters)\index{fences}\index{delimiters}}
\begin{table}[H]
\centering
\small
\begin{tabular}{|ccc|}
\hline
\textbf{Description}                             & \textbf{Command}                           & \textbf{Output}    \\                              
Round Brackets & \texttt{(x)} & $(x)$ \\ 
Square Brackets & \texttt{[x]} & $[x]$ \\ 
Curly Brackets & \command{\{}\texttt{x}\command{\}} & $\{x\}$ \\ 
Angled Brackets & \command{langle}\texttt{ x }\command{rangle} & $\langle x \rangle$ \\ 
Floor & \command{lfloor}\texttt{ x }\command{rfloor} & $\lfloor x \rfloor$ \\ 
Ceiling & \command{lceil}\texttt{ x }\command{rceil} & $\lceil x \rceil$ \\ 
Norm & \command{|}\texttt{x}\command{|} & $\|x\|$ \\ 
Upper Corners & \command{ulcorner}\texttt{ x }\command{urcorner} & $\ulcorner x \urcorner$ \\ 
Lower Corners & \command{llcorner}\texttt{ x }\command{lrcorner} & $\llcorner x \lrcorner$ \\ 
\hline
\end{tabular}
\end{table}
\index{brackets}\index{parentheses}\index{round brackets}
\index{square brackets}
\index{curly brackets}
\index{floor}
\index{ceiling}
\index{norm}\index{double bars}
\index{corners}
\noindent If the fences are too small, use \command{left} and \command{right} before the left and right fence, respectively.  
\begin{LTXexample}[style=myLaTeX, pos=o]
\[n \to \infty \implies \left( 1+\frac{1}{n} \right)^{n} \to e\]
\end{LTXexample}
$\\$
For bigger, custom fence sizes, use \command{big}, \command{Big}, \command{bigg}, or \command{Bigg} before both the left and right fence.

% Dots %
\subsection{Dots\index{dots}}
\begin{table}[H]
\centering
\small
\begin{tabular}{|ccc|}
\hline
\textbf{Description}                             & \textbf{Command}                           & \textbf{Output}    \\                              
Comma Separated List & \texttt{1, 2, \command{dotsc}, 5}  & $1, 2, \dotsc, 5$ \\ 
Lower Dots & \texttt{1, 2, \command{ldots}, 5}  & $1, 2, \ldots, 5$ \\ 
Multiplication Dots & \texttt{1 \command{cdot} 2 \command{cdot} 3 \command{dotsm} 5}  & $1 \cdot 2 \cdot 3 \dotsm 5$ \\ 
Binary Operation Dots & \texttt{1 + 2 + \command{dotsb} + 5}  & $1 + 2 + \dotsb + 5$ \\ 
Other Dots & \texttt{1,  2, \command{dotso}, 5}  & $1, 2, \dotso, 5$ \\ 
Vertical Dots & \command{vdots} & $\vdots$ \\
Diagonal Dots & \command{ddots} & $\ddots$ \\
\hline
\end{tabular}
\end{table}
\index{vertical dots}
\index{diagonal dots}
\index{centered dots}
\index{multiplication dots}
\index{comma separated list}

% Arrows %
\subsection{Arrows\index{arrows}}
\begin{table}[H]
\centering
\small
\begin{tabular}{|ccc|}
\hline
\textbf{Description}                             & \textbf{Command}                           & \textbf{Output}    \\                              
Right Arrow & \command{to}, \command{rightarrow} & $\to$ \\ 
Long Right Arrow & \command{longrightarrow} & $\longrightarrow$ \\ 
Not Right Arrow & \command{nrightarrow} & $\nrightarrow$ \\ 
Thick Right Arrow & \command{Rightarrow} & $\Rightarrow$ \\ 
Thick Long Right Arrow & \command{Longrightarrow} & $\Longrightarrow$ \\ 
Thick Not Right Arrow & \command{nRightarrow} & $\nRightarrow$ \\ 
Left Arrow & \command{leftarrow}, \command{gets} & $\leftarrow$ \\ 
Long Left Arrow & \command{longleftarrow} & $\longleftarrow$ \\ 
Not Left Arrow & \command{nleftarrow} & $\nleftarrow$ \\ 
Thick Left Arrow & \command{Leftarrow} & $\Leftarrow$ \\ 
Thick Long Left Arrow & \command{Longleftarrow} & $\Longleftarrow$ \\ 
Thick Not Left Arrow & \command{nLeftarrow} & $\nLeftarrow$ \\ 
Left-Right Arrow & \command{leftrightarrow} & $\leftrightarrow$ \\ 
Not Left-Right Arrow & \command{nleftrightarrow} & $\nleftrightarrow$ \\ 
Thick Left-Right Arrow & \command{iff} & $\iff$ \\ 
Up Arrow & \command{uparrow} & $\uparrow$ \\ 
Thick Up Arrow & \command{Uparrow} & $\Uparrow$ \\ 
Down Arrow & \command{downarrow} & $\downarrow$ \\ 
Thick Down Arrow & \command{Downarrow} & $\Downarrow$ \\ 
Up-Down Arrow & \command{updownarrow} & $\updownarrow$ \\ 
Thick Up-Down Arrow & \command{Updownarrow} & $\Updownarrow$ \\ 
Maps To & \command{mapsto} & $\mapsto$ \\ 
Maps To (Long) & \command{longmapsto} & $\longmapsto$ \\ 
Leads To & \command{leadsto} & $\leadsto$ \\ 
\hline
\end{tabular}
\end{table}
\index{right arrow}
\index{left arrow}
\index{up arrow}
\index{down arrow}
\index{thick arrows}
\index{maps}\index{leads}
\index{iff}\index{implies}\index{gets}
\index{negated arrows}

% Decorations %
\subsection{Decorations\index{decorations}}
\begin{table}[H]
\centering
\small
\begin{tabular}{|ccc|}
\hline
\textbf{Description}                             & \textbf{Command}                           & \textbf{Output}    \\    
Over-brace & \command{overbrace}\texttt{\{x+y\}\^{}\{=y+x\}} & $\overbrace{x+y}^{=y+x}$\\                           
Prime & \texttt{f'}  & $f'$ \\ 
Prime Prime & \texttt{f''}  & $f''$ \\ 
Dot & \command{dot}\texttt{\{x\}}  & $\dot{x}$ \\ 
Dot Dot & \command{ddot}\texttt{\{x\}}  & $\ddot{x}$ \\ 
Hat & \command{hat}\texttt{\{x\}}  & $\hat{x}$ \\ 
Wide Hat & \command{widehat}\texttt{\{x+y\}}  & $\widehat{x+y}$ \\ 
Tilde & \command{tilde}\texttt{\{x\}}  & $\tilde{x}$ \\ 
Wide Tilde & \command{widetilde}\texttt{\{x+y\}}  & $\widetilde{x+y}$ \\ 
Bar & \command{bar}\texttt{\{x\}}  & $\bar{x}$ \\ 
Under-brace & \command{underbrace}\texttt{\{x+y\}\_{}\{=y+x\}} & $\underbrace{x+y}_{=y+x}$ \\ 
\hline
\end{tabular}
\end{table}
\index{prime}
\index{derivative}\index{second derivative}
\index{hat}
\index{tilde}
\index{bar}
\index{star}
\index{over-brace}
\index{under-brace}

% Misc %
\subsection{Miscellaneous\index{miscellaneous symbols}}
\begin{table}[H]
\centering
\small
\begin{tabular}{|ccc|}
\hline
\textbf{Description}                             & \textbf{Command}                           & \textbf{Output}    \\    
Asterisk & \command{ast} & $\ast$ \\ 
Bow Tie & \command{bowtie} & $\bowtie$ \\ 
Bullet & \command{bullet} & $\bullet$ \\ 
Dagger & \command{dagger} & $\dagger$ \\ 
Curly l & \command{ell} & $\ell$ \\ 
Star & \command{star} & $\star$ \\ 
Surd & \command{surd} & $\surd$ \\ 
Tick & \command{checkmark} & $\checkmark$ \\ 
Tilde & \command{sim} & $\sim$ \\ 
\hline
\end{tabular}
\end{table}
\index{star}
\index{asterisk}
\index{bow-tie}
\index{dagger}
\index{curly l}
\index{bullet}
\index{surd}
\index{tick}
\index{tilde}

% Sets %
\section{Set Theory\index{set theory}}

% Number Sets %
\subsection{Number Sets\index{sets}\index{number sets}}
\begin{table}[H]
\centering
\small
\begin{tabular}{|ccc|}
\hline
\textbf{Description}                             & \textbf{Command}                           & \textbf{Output}    \\            
Boolean Numbers & \command{mathbb}\texttt{\{B\}} & $\mathbb{B}$ \\ 
Prime Numbers & \command{mathbb}\texttt{\{P\}} & $\mathbb{P}$ \\ 
Natural Numbers & \command{mathbb}\texttt{\{N\}} & $\mathbb{N}$ \\ 
Whole Numbers & \command{mathbb}\texttt{\{W\}} & $\mathbb{W}$ \\ 
Integers & \command{mathbb}\texttt{\{Z\}} & $\mathbb{Z}$ \\ 
Rationals & \command{mathbb}\texttt{\{Q\}} & $\mathbb{Q}$ \\ 
Algebraic Numbers & \command{mathbb}\texttt{\{A\}} & $\mathbb{A}$ \\ 
Irrationals & \command{mathbb}\texttt{\{I\}} & $\mathbb{I}$ \\ 
Reals & \command{mathbb}\texttt{\{R\}} & $\mathbb{R}$ \\ 
Complex Numbers & \command{mathbb}\texttt{\{C\}} & $\mathbb{C}$ \\ 
Quaternions & \command{mathbb}\texttt{\{H\}} & $\mathbb{H}$ \\ 
Octonions & \command{mathbb}\texttt{\{O\}} & $\mathbb{O}$ \\ 
Sedenions & \command{mathbb}\texttt{\{S\}} & $\mathbb{S}$ \\ 
Empty Set & \command{emptyset}, \command{varnothing} & $\emptyset$,  $\varnothing$ \\               
Power Set & \command{mathcal}\texttt{\{P\}} & $\mathcal{P}$ \\ 
\hline
\end{tabular}
\end{table}
\index{boolean numbers}
\index{prime numbers}
\index{natural numbers}
\index{whole numbers}
\index{integers}
\index{rationals}\index{fractions}
\index{algebraic numbers}
\index{irrationals}
\index{reals}
\index{complex numbers}
\index{quaternions}
\index{octonions}
\index{sedenions}
\index{empty set}
\index{power set}

% Set Notation %
\subsection{Set Notation\index{set notation}}
\begin{table}[H]
\centering
\small
\begin{tabular}{|ccc|}
\hline
\textbf{Description}                             & \textbf{Command}                           & \textbf{Output}    \\                           
Brackets & \texttt{\command{\{}3, 1, 4\command{\}}} & $\{3, 1, 4\}$ \\ 
Cardinality & \command{mathbf}\texttt{\{card\}(S)}, \texttt{|S|}& $\mathbf{card}(S)$, $|S|$ \\ 
Definition & \texttt{A:=B} & $A:=B$ \\ 
Element of & \command{in} & $\in$ \\ 
Not an Element of & \command{notin} & $\notin$ \\ 
Subset of & \command{subset} & $\subset$ \\ 
Subset of & \command{subseteq} & $\subseteq$ \\ 
Subset of but Not Equal to & \command{subsetneq} & $\subsetneq$ \\ 
Not a Subset of & \command{not}\command{subset} & $\not\subset$ \\ 
Not a Subset of & \command{nsubseteq} & $\nsubseteq$ \\ 
Contains & \command{supset} & $\supset$ \\ 
Contains & \command{supseteq} & $\supseteq$ \\ 
Union & \command{cup} & $\cup$ \\ 
Big Union & \command{bigcup}\texttt{\_{}\{n=1\}\^{}\{10\}\{A\_{}n\}}& $\bigcup_{n=1}^{10}{A_n}$ \\ 
Disjoint Union & \command{sqcup} & $\sqcup$ \\ 
Intersection & \command{cap} & $\cap$ \\ 
Big Intersection & \command{bigcap}\texttt{\_{}\{n=1\}\^{}\{10\}\{A\_{}n\}}& $\bigcap_{n=1}^{10}{A_n}$ \\ 
Set Difference & \command{setminus} & $\setminus$ \\ 
Symmetric Difference & \command{triangle} & $\triangle$ \\ 
Set Complement & \texttt{A\^{}\{mathsf\{c\}\}} & $A^{\mathsf{c}}$ \\ 
Set Complement & \command{overline}\texttt{\{A\}} & $\overline{A}$ \\ 
Cartesian Product & \command{times} & $\times$ \\ 
\hline
\end{tabular}
\end{table}
\index{element}\index{member}
\index{subset}\index{contained in}
\index{superset}\index{contains}
\index{union}
\index{intersection}
\index{set difference}
\index{set complement}
\index{complement}

% Braket Package %
\subsection{\texttt{braket} package\index{braket}}
You can load the \texttt{braket} package when typesetting sets.
\begin{LTXexample}[style=myLaTeX, pos=o]
% Preamble 
\usepackage{braket}

% Body 
\[ \Set{x \in \mathbb{R} | 0 < x < \frac{1}{3}}\]
\end{LTXexample}

% Logic %
\section{Logic\index{logic}}
\begin{table}[H]
\centering
\small
\begin{tabular}{|ccc|}
\hline
\textbf{Description}                             & \textbf{Command}                           & \textbf{Output}    \\                              
Not & \command{neg}, \command{sim} & $\neg$, $\sim$ \\ 
And & \command{land} & $\land$ \\ 
Or & \command{lor} & $\lor$ \\ 
Exclusive Or (XOR) & \command{oplus} & $\oplus$ \\ 
If$\dots$Then & \command{implies}, \command{Longrightarrow} & $\Longrightarrow$ \\ 
Only If & \command{Longleftarrow} & $\Longleftarrow$ \\ 
If and Only If & \command{iff} & $\iff$ \\ 
Equivalence & \command{equiv} & $\equiv$ \\ 
Therefore & \command{therefore} & $\therefore$ \\ 
Because & \command{because} & $\because$ \\ 
Exists & \command{exists} & $\exists$ \\ 
Exists Uniquely & \command{exists!} & $\exists!$ \\ 
There is No & \command{nexists} & $\nexists$ \\ 
For All & \command{for all} & $\forall$ \\ 
Top & \command{top} & $\top$ \\ 
Bottom & \command{bot} & $\bot$ \\ 
\hline
\end{tabular}
\end{table}
\index{negation}\index{not}
\index{and}
\index{or}
\index{xor}\index{exclusive or}
\index{if-then}\index{implies}
\index{only if}\index{implied by}
\index{if and only if}
\index{equivalence}
\index{exists}
\index{for all}
\index{top}
\index{bottom}
\index{therefore}
\index{because}
\noindent More logic symbols can be found on \href{https://en.wikipedia.org/wiki/List_of_logic_symbols}{Wikipedia}.

% Algebra %
\section{Algebra\index{algebra}}

% Infinity %
\subsection{Infinity\index{infinity}}
First, you ought to know the commands for $\infty$ and $-\infty$.
\begin{table}[H]
\centering
\small
\begin{tabular}{|ccc|}
\hline
\textbf{Description}                             & \textbf{Command}                           & \textbf{Output}    \\                              
Infinity & \command{infty} & $\infty$ \\ 
Negative Infinity & \texttt{-}\command{infty} & $-\infty$ \\ 
\hline
\end{tabular}
\end{table}
\index{infinity}
\index{negative infinity}

% Intervals %
\subsection{Intervals\index{intervals}}
There are 9 types of intervals:
\begin{table}[H]
\centering
\small
\begin{tabular}{|ccc|}
\hline
\textbf{Description}                             & \textbf{Command}                           & \textbf{Output}    \\                              
Finite Open & \texttt{(a, b)} & $(a, b)$ \\ 
Finite Closed & \texttt{[a, b]} & $[a, b]$ \\ 
Finite Half Closed - Half Open & \texttt{[a, b)} & $[a, b)$ \\ 
Infinite Half Closed - Half Open & \texttt{[a, \command{infty})} & $[a, \infty)$ \\ 
Infinite Half Open - Half Closed & \texttt{(-\command{infty}, b]} & $(-\infty, b]$ \\ 
Infinite Open & \texttt{(a, \command{infty})} & $(a, \infty)$ \\ 
Infinite Open & \texttt{(-\command{infty}, b)} & $(-\infty, b)$ \\ 
Reals & \texttt{(\command{infty}, -\command{infty})} & $(\infty, -\infty)$ \\ 
\hline
\end{tabular}
\end{table}
\index{closed interval}
\index{open interval}
\index{half open - half closed}
\index{half closed - half open}
\index{reals}

% Functions %
\subsection{Functions\index{functions}\index{maps}}
\begin{table}[H]
\centering
\small
\begin{tabular}{|ccc|}
\hline
\textbf{Description}                             & \textbf{Command}                           & \textbf{Output}    \\                              
Colon & \command{colon}& $\colon$ \\ 
Function & \command{to}, \command{rightarrow} & $\to$ \\ 
Maps To & \command{mapsto}& $\mapsto$ \\ 
Injection & \command{rightarrowtail} & $\rightarrowtail$ \\ 
Injection & \command{xhookrightarrow\{\}} & $\xhookrightarrow{}$ \\ 
Injection & \command{xrightarrow}\texttt{\{\command{tiny} 1:1\}} & $\xrightarrow{\tiny 1:1}$ \\ 
Injection & \command{xrightarrow}\texttt{[\command{tiny} 1:1]\{\}}  & $\xrightarrow[\tiny 1:1]{}$ \\ 
Surjection & \command{twoheadrightarrow} & $\twoheadrightarrow$ \\ 
Surjection & \command{xrightarrow}\texttt{\{\command{tiny} \command{text}\{onto\}\}} & $\xrightarrow{\tiny \text{onto}}$ \\ 
Bijection & \command{xrightarrow}\texttt{\{\command{tiny} 1:1, \command{text}\{ onto\}\}}  & $\xrightarrow{\tiny 1:1, \text{ onto}}$ \\ 
Bijection & \command{xrightarrow}\texttt{\{\command{tiny}\command{text}\{bij\}\}}  & $\xrightarrow{\tiny \text{bij}}$ \\ 
Composition & \command{circ}& $\circ$ \\ 
Restriction & \texttt{f|\_{}\{X\}} & $f|_{X}$ \\ 
Inverse & \texttt{f\^{}\{-1\}} & $f^{-1}$ \\ 
Convolution &  \command{ast} & $\ast$ \\ 
Fourier Transform &  \command{hat}\texttt{\{f\}}& $\hat{f}$ \\ 
\hline
\end{tabular}
\end{table}
\index{function}
\index{map}
\index{injection}
\index{surjection}
\index{bijection}
\index{colon}
\index{composition}
\noindent While you could type out \say{:}, \command{colon} allows for proper spacing.
\begin{LTXexample}[style=myLaTeX, pos=o]
% :
$f \circ g: [0, 1] \to [0, 1]$ is a function. \\

% \colon
$f \circ g \circ h \colon A \to B$ is a function.
\end{LTXexample}
$\\$
Use the \texttt{cases} environment\index{environments!cases@\texttt{cases}} to define a piecewise function\index{piecewise function}.
\begin{LTXexample}[style=myLaTeX, pos=o]
\[ 
f(x) = 1_{\mathbb{Q}}(x) = 
\begin{cases}
1 & x \in \mathbb{Q} \\
0 & x \notin \mathbb{Q}
\end{cases} 
\]
\end{LTXexample}
$\\$
If you need to include text, then:
\begin{LTXexample}[style=myLaTeX, pos=o]
\[ 
f(x) = 1_{\mathbb{Q}}(x) = 
\begin{cases}
1 & \text{if $x \in \mathbb{Q}$} \\
0 & \text{if $x \notin \mathbb{Q}$}
\end{cases} 
\]
\end{LTXexample}

% Geometry %
\section{Geometry\index{geometry}}

% Geometry Notation %
\subsection{Geometry Notation}
\begin{table}[H]
\centering
\small
\begin{tabular}{|ccc|}
\hline
\textbf{Description}                             & \textbf{Command}                           & \textbf{Output}    \\                              
Line Segment & \command{overline}\texttt{\{AB\}}& $\overline{AB}$ \\ 
Ray & \command{overrightarrow}\texttt{\{AB\}}& $\overrightarrow{AB}$ \\ 
Line & \command{overleftrightarrow}\texttt{\{AB\}}& $\overleftrightarrow{AB}$ \\ 
Triangle & \command{triangle}\texttt{\{ABC\}} & $\triangle{ABC}$ \\
Square & \command{square}\texttt{\{ABC\}} & $\square{ABC}$ \\
Angle & \command{angle}\texttt{\{ABC\}} & $\angle{ABC}$ \\
Measured Angle & \command{measuredangle}\texttt{\{ABC\}} & $\measuredangle{ABC}$ \\
Degrees & \texttt{180\^{}\{\command{circ}\}} & $180^{\circ}$ \\
Congruent & \command{cong} & $\cong$ \\
Not Congruent & \command{ncong} & $\ncong$ \\
Similar & \command{sim} & $\sim$ \\
Not Similar & \command{nsim} & $\nsim$ \\
Parallel & \command{parallel} & $\parallel$ \\
Not Parallel & \command{nparallel} & $\nparallel$ \\
Perpendicular & \command{perp} & $\perp$ \\
Not Perpendicular & \command{not}\command{perp} & $\not\perp$ \\
\hline
\end{tabular}
\end{table}
\index{line}
\index{segment}
\index{ray}
\index{triangle}
\index{square}
\index{angle}
\index{degrees}
\index{congruent}
\index{not congruent}
\index{similar}
\index{not similar}
\index{parallel}
\index{not parallel}
\index{perpendicular}
\index{not perpendicular}

% Trig + Hyperbolic Functions %
\subsection{Trigonometry \& Hyperbolic Functions\index{trigonometry}\index{hyperbolic functions}}
\begin{table}[H]
\centering
\small
\begin{tabular}{|ccc|}
\hline
\textbf{Description}                             & \textbf{Command}                           & \textbf{Output}    \\                              
Sine & \command{sin}\texttt{\{\textbackslash pi\}}, \command{sin}\texttt{(\textbackslash pi)} & $\sin{\pi}$,  $\sin(\pi)$ \\ 
Cosine & \command{cos}\texttt{\{\textbackslash pi\}} & $\cos{\pi}$ \\ 
Tangent & \command{tan}\texttt{\{\textbackslash pi\}} & $\tan{\pi}$ \\ 
Cosecant & \command{csc}\texttt{\{\textbackslash pi\}} & $\csc{\pi}$ \\ 
Secant & \command{sec}\texttt{\{\textbackslash pi\}} & $\sec{\pi}$ \\ 
Cotangent & \command{cot}\texttt{\{\textbackslash pi\}} & $\cot{\pi}$ \\ 
Inverse Sine & \command{arcsin}\texttt{\{0\}} & $\arcsin{0}$ \\ 
Inverse Cosine & \command{arccos}\texttt{\{0\}} & $\arccos{0}$ \\ 
Inverse Tangent & \command{arctan}\texttt{\{0\}} & $\arctan{0}$ \\ 
Hyperbolic Sine & \command{sinh}\texttt{\{0\}} & $\sinh{0}$ \\ 
Hyperbolic Cosine & \command{cosh}\texttt{\{0\}} & $\cosh{0}$ \\ 
Hyperbolic Tangent & \command{tanh}\texttt{\{0\}} & $\tanh{0}$ \\ 
\hline
\end{tabular}
\end{table}
\index{sine}
\index{cosine}
\index{tangent}
\index{cosecant}
\index{secant}
\index{cotangent}
\index{arcsin}\index{inverse sine}
\index{arccos}\index{inverse cosine}
\index{arctan}\index{inverse tangent}
\index{sinh}
\index{cosh}
\index{tanh}

% Sums %
\subsection{Sums\index{sums}}
\label{sums}
Sums are different in inline and display mode.
\begin{LTXexample}[style=myLaTeX, pos=o]
The harmonic series 
$\sum_{n=1}^{\infty}{\frac{1}{n}}$ 
is divergent. \\

The harmonic series 
\[\sum_{n=1}^{\infty}{\frac{1}{n}}\] 
is divergent.
\end{LTXexample} 
$\\$
While the curly braces \texttt{\{\}} are not necessary, they make the code readable.
\begin{LTXexample}[style=myLaTeX, pos=o]
\[\sum_a^b \frac{1}{n}\] 
\end{LTXexample} 
$\\$
You can also typeset double sums.
\begin{LTXexample}[style=myLaTeX, pos=o]
\[\sum_{i=1}^{2}{\sum_{j=1}^{2}{i+j}} = 12\]
\end{LTXexample} 
$\\$
Use \command{substack} to write the limits over multiple lines.
\begin{LTXexample}[style=myLaTeX, pos=o]
\[\sum_{
\substack{
0 \leq i \leq 2 \\
0 \leq j \leq 2
}
}^{}{i+j}=12\]
\end{LTXexample} 
You can forcefully change the position of the limits for sums using \command{limits} and \command{nolimits}\footnote{This also applies for products, integrals, and limits.}.
\begin{LTXexample}[style=myLaTeX, pos=o]
% Inline Mode
$\sum_{n=1}^{5}{n}$

% Inline Mode (placing limit position under sum)
$\sum\limits_{n=1}^{5}{n}$

% Inline Mode (placing limit position besides sum)
$\sum\nolimits_{n=1}^{5}{n}$

% Display Mode
\[\sum_{n=1}^{5}{n}\] 

% Display Mode (placing limit position under sum)
\[\sum\limits_{n=1}^{5}{n}\] 

% Display Mode (placing limit position besides sum)
\[\sum\nolimits_{n=1}^{5}{n}\] 
\end{LTXexample} 
% Products %
\subsection{Products\index{product}\index{multiplication}}
Refer to \ref{sums} on page \pageref{sums} and replace \texttt{sum} with \texttt{prod}.
\begin{LTXexample}[style=myLaTeX, pos=o]
% Inline Mode
$\prod_{n=1}^{50}{n}=50!$ \\

% Display Mode
The product \[\prod_{n=1}^{50}{n}\]
$=50!$
\end{LTXexample} 
\indexcommand{prod}

% Caclulus % 
\section{Calculus\index{calculus}}

% Derivatives %
\subsection{Derivatives\index{derivatives}\index{differentiation}\index{partial derivatives}\index{ordinary derivatives}\index{higher-order derivatives}}
You can write a derivative as follows:
\begin{LTXexample}[style=myLaTeX, pos=o]
% Leibniz Notation
If $f(x)=x^2$, then
\[\frac{df}{dx}=2x.\]

% Lagrange Notation
Using other notation:
\[f'(x)=2x\]
\end{LTXexample} 
$\\$
Notice the slant in $df$. For an upright $\mathrm{d}$, type the following in the preamble:
\begin{lstlisting}[style=myLaTeX, columns=fullflexible]
\newcommand{\dee}{\mathrm{d}}
\end{lstlisting} 
$\\$
\command{dee} is my choice, so you can use something else.
\begin{LTXexample}[style=myLaTeX, pos=o]
\[\frac{\dee f}{\dee x} = 2x\]
\end{LTXexample} 
$\\$
If you need to evaluate derivatives:
\begin{LTXexample}[style=myLaTeX, pos=o]
\[\left.\frac{\dee f}{\dee x}\right|_{x=2}=4\]
\end{LTXexample} 
\index{evaluate derivatives}
$\\$
Partial derivatives are typeset using \command{partial}.
\begin{LTXexample}[style=myLaTeX, pos=o]
\[\frac{\partial g}{\partial x \partial y}\]
\end{LTXexample} 

% diffcoeff Package %
\subsubsection{\texttt{diffcoeff} package}
\texttt{diffcoeff} with the \texttt{ISO} option also takes care of the upright $\mathrm{d}$. It is also handy for higher-order and partial derivatives.
\begin{lstlisting}[style=myLaTeX, columns=fullflexible]
 % Preamble
\uspackage[ISO]{diffcoeff}
\end{lstlisting} 
$\\$
Typesetting ordinary derivatives:
\begin{LTXexample}[style=myLaTeX, pos=o]
\[\diff{f}{x}\]
\[\diff{f}/{x}\]
\[\diff[n]{f}{x}\]
\[\diff[n]{f}/{x}\]
\[\diff{\cos(\sin x)}{(\sin x)}\]
\[\diff[n]{\cos(\sin{x})}{\sin{x}}\]
\[\diff*{f(x)}{x}\]
\[\diff*{\diff{y}{x}}{x}\]
\[\diff[n]{f}{x}[x=0]\]
\end{LTXexample} 
$\\$
Typesetting partial derivatives:
\begin{LTXexample}[style=myLaTeX, pos=o]
\[\diffp{f}{x}\]
\[\diffp{f}/{x}\]
\[\diffp[n]{f}{x}\]
\[\diffp[n]{f}/{x}\]
\[\diffp[n]{f(x,y)}{x}[(0,0)]\]
\[\diffp{f}{x, y, z}\]
\[\diffp[2, 3, 4, 1]{f(x, y, z, w)}{x, y, z, w}\]
\end{LTXexample} 
$\\$
More package information can be found on \href{https://ctan.org/pkg/diffcoeff?lang=en}{CTAN}.

% Integrals %
\subsection{Integration\index{integration}\index{antiderivatives}}
\label{integration}
Refer to \ref{sums} on page \pageref{sums} and replace \texttt{sum} with \texttt{int}. To include the differential, add \texttt{\textbackslash,\textbackslash dee x}. 
\begin{LTXexample}[style=myLaTeX, pos=o]
The integral $\int_{0}^{\infty}{e^x} \,\dee x$ diverges.
\end{LTXexample} 
\begin{LTXexample}[style=myLaTeX, pos=o]
\[\int_{0}^{2}{2x} \,\dee x = \left[x^2\right]^{2}_{0} = 4\]
\end{LTXexample} 
\index{differentials}
\index{integral spacing}
\index{spacing in integral}
$\\$
For multiple integrals, use \command{int} multiple times. 
\begin{LTXexample}[style=myLaTeX, pos=o]
If $I_1=I_2=[0, 2]$, then
\[\int_{I_1}{\int_{I_2}{xy \,\dee x} \,\dee y} = 4\]

Explicitly:
\[\int_{0}^{2}{\int_{0}^{2}{xy \,\dee x} \,\dee y} = 4\]

\end{LTXexample}
$\\$
Different types of integrals:
\begin{LTXexample}[style=myLaTeX, pos=o]
% Integral with Specified Limits (Besides Integral)
\[\int_{-\infty}^{\infty} f = 0\]

% Integral with Specified Limits (Under Integral)
\[\int\limits_{-\infty}^{\infty} f = 0\]

% Double / Surface Integral
\[\iint_A f = F\]

% Triple / Volume Integral
\[\iiint_V f = F\]

% Quadruple Integral
\[\iiiint_V f = F\]

% Multiple Integral
\[\idotsint_V f = F\]

% Line Integral
\[\oint_V f = F\]

\end{LTXexample}
\indexcommand{iint}
\indexcommand{iiint}
\indexcommand{iiint}
\indexcommand{idotsint}
\indexcommand{oint}
\index{double integral}\index{surface integral}
\index{triple integral}\index{volume integral}
\index{quadruple integral}\index{multiple integral}
\index{line integral}\index{curve integral}\index{closed integral}

% Multivariable Calculus %
\subsection{Multivariable Calculus\index{vector calculus}}
\begin{table}[H]
\centering
\small
\begin{tabular}{|ccc|}
\hline
\textbf{Description}                             & \textbf{Command}                           & \textbf{Output}    \\                              
Gradient & \command{nabla}\texttt{\{f\}} & $\nabla{f}$ \\ 
Divergence & \command{nabla}\command{cdot}\texttt{\{F\}} & $\nabla\cdot{F}$ \\ 
Divergence & \command{nabla}\command{times}\texttt{\{f\}} & $\nabla\times{F}$ \\ 
Laplace Operator & \command{Delta}\texttt{\{f\}} & $\Delta{f}$ \\ 
D'Alembert Operator & \command{square}\texttt{\{f\}} & $\square{f}$ \\ 
\hline
\end{tabular}
\end{table}
\index{gradient}
\index{divergence}
\index{curl}
\index{Laplace operator}
\index{D'Alembert operator}

% Analysis % 
\section{Analysis\index{analysis}}

% Sequences and Series %
\subsection{Sequences\index{sequences}}
Use \texttt{()} to denote sequences. 
\begin{LTXexample}[style=myLaTeX, pos=o]
Let $(a_{n})=(1, 2, 3, 4, 5, \ldots)$ be a sequence. 
Then \[n \to \infty \implies a_{n} \to \infty.\]
\end{LTXexample}

% Limits %
\subsection{Limits\index{limits}}
Limits can also be typeset easily.
\begin{LTXexample}[style=myLaTeX, pos=o]
If the limit of $f(x)$ exists at $x=a$, then $(\forall\varepsilon>0) (\exists\delta>0) (0<|x-a|<\delta \implies |f(x)-f(a)|<\varepsilon)$. \\
\end{LTXexample}
\begin{LTXexample}[style=myLaTeX, pos=o]
% Inline Mode
$\lim_{n \to \infty}{\frac{1}{n}} = 0$ \\

$\lim_{n \to 2^{+}}{\frac{1}{n}} = \frac{1}{2}$ \\

% Display Mode
\[\lim_{n \to \infty}{\frac{1}{n}} = 0\]

\[f'(x) = \lim_{h \to 0^{+}}{\frac{f(x+h) - f(x)}{h}}\] \\
\end{LTXexample} 
$\\$
\command{substack} (refer to \ref{sums} on page \pageref{sums}) can also be applied to limits.

% Inf + Sup %
\subsection{Infimum \& Supremum\index{infimum}\index{supremum}}
For limit inferior and superior, replace \command{lim} with \command{liminf} and \command{limsup}, respectively. 
\begin{LTXexample}[style=myLaTeX, pos=o]
% Limit Superior
\[\limsup_{n \to \infty}{x_{n}} = 1\] 
\[\varlimsup_{n \to \infty}{x_{n}} = 1\] 

% Limit Inferior
\[\liminf_{n \to \infty}{x_{n}} = -1\] 
\[\varliminf_{n \to \infty}{x_{n}} = -1\] 
\end{LTXexample} 
$\\$
Other important commands include:
\begin{table}[H]
\centering
\small
\begin{tabular}{|ccc|}
\hline
\textbf{Description}                             & \textbf{Command}                           & \textbf{Output}    \\                              
Minimum & \command{min}\texttt{\{A\}} & $\min{A}$ \\ 
Maximum & \command{max}\texttt{\{A\}} & $\max{A}$ \\ 
Infimum & \command{inf}\texttt{\{A\}} & $\inf{A}$ \\ 
Supremum & \command{sup}\texttt{\{A\}} & $\sup{A}$ \\ 
\hline
\end{tabular}
\end{table}
\index{minimum}
\index{maximum}
\index{infimum}\index{limit inferior}
\index{supremum}\index{limit superior}

% Big O Notation %
\subsection{Big O Notation\index{big O notation}}
\begin{table}[H]
\centering
\small
\begin{tabular}{|ccc|}
\hline
\textbf{Description}                             & \textbf{Command}                           & \textbf{Output}    \\                              
Small o & \texttt{o(g)} & $o(g)$ \\ 
Big O & \command{mathcal}\texttt{\{O\}(g)} & $\mathcal{O}(g)$ \\ 
Big Theta & \command{Theta}\texttt{(g)} & $\Theta(g)$ \\ 
Big Omega & \command{Omega}\texttt{(g)} & $\Omega(g)$ \\ 
Small omega & \command{omega}\texttt{(g)} & $\omega(g)$ \\
\hline
\end{tabular}
\end{table}
\index{small o}
\index{little o}
\index{big Omega}
\index{small omega}
\index{big Theta}

% Algebra %
\section{Abstract Algebra\index{abstract algebra}\index{algebra}}

% Equivalence Relations & Classes %
\subsection{Equivalence Classes \& Relations\index{equivalence classes}\index{equivalence relations}}
\begin{table}[H]
\centering
\small
\begin{tabular}{|ccc|}
\hline
\textbf{Description}                             & \textbf{Command}                           & \textbf{Output}    \\                              
Equivalence Class & \texttt{[a]} & $[a]$ \\ 
Equivalence Relation & \command{sim} & $\sim$ \\ 
Equivalence Relation & \command{backsim} & $\backsim$ \\ 
\hline
\end{tabular}
\end{table}

% Group Theory %
\subsection{Group Theory\index{group}}
\begin{table}[H]
\centering
\small
\begin{tabular}{|ccc|}
\hline
\textbf{Description}                             & \textbf{Command}                           & \textbf{Output}    \\                              
Group Isomorphism & \command{simeq} & $\simeq$ \\ 
Direct Product & \command{times} & $\times$ \\ 
Semi-Direct Product & \command{rtimes} & $\rtimes$ \\ 
Wreath Product & \command{wr} & $\wr$ \\ 
Subgroup & \command{leq} & $\leq$ \\ 
Normal Subgroup & \command{vartriangleleft} & $\vartriangleleft$ \\ 
Not a Normal Subgroup & \command{not}\command{vartriangleleft} & $\not\vartriangleleft$ \\ 
Quotient Group & \texttt{G / H} & $G / H$ \\ 
Index of a Subgroup & \texttt{[G : H]} & $ [G : H]$ \\ 
Generator & \command{langle}\texttt{ X }\command{rangle}  & $\langle X \rangle$ \\ 
Commutator & \texttt{[g, h]} & $[g, h]$ \\ 
\hline
\end{tabular}
\end{table}
\index{isomorphism}
\index{direct product}
\index{semi-direct product}
\index{Wreath product}
\index{subgroup}
\index{normal subgroup}
\index{index}
\index{generator}
\index{commutator}

% Field Theory %
\subsection{Field Theory\index{field}}
\begin{table}[H]
\centering
\small
\begin{tabular}{|ccc|}
\hline
\textbf{Description}                             & \textbf{Command}                           & \textbf{Output}    \\                              
Field Extension & \texttt{L : K} & $L : K$ \\ 
Degree of Field Extension & \texttt{[L : K]} & $ [L : K]$ \\ 
Algebraic Closure & \command{overline}\texttt{\{K\}} & $\overline{K}$ \\ 
\hline
\end{tabular}
\end{table}
\index{field extension}
\index{algebraic closure}

% Discrete Mathematics %
\section{Discrete Mathematics \index{discrete mathematics}}

% Number Theory %
\subsection{Number Theory}
\begin{table}[H]
\centering
\small
\begin{tabular}{|ccc|}
\hline
\textbf{Description}                             & \textbf{Command}                           & \textbf{Output}    \\                              
Divides & \texttt{a \command{mid} b} & $a \mid b$ \\ 
Does Not Divide  & \texttt{a \command{nmid} b} & $a \nmid b$ \\ 
Congruence With () & \texttt{a \command{equiv} b \command{pmod}\{n\}} & $a \equiv b \pmod{n}$ \\ 
Congruence Without () & \texttt{a \command{equiv} b \command{mod}\{n\}} & $a \equiv b \mod{n}$ \\ 
Greatest Common Divisor & \texttt{\command{gcd}(100, 10)} & $\gcd(100, 10)$ \\
Euler's Totient Function & \texttt{\command{phi}(n)} & $\phi(n)$ \\
\hline
\end{tabular}
\end{table}
\index{divides}
\index{congruence}\index{modular arithmetic}
\index{greatest common divisor}
\index{totient function}\index{Euler's totient function}

% Continued Fractions %
\subsection{Continued Fractions\index{continued fractions}}
\command{cfrac} does the job. The options \texttt{[r]} or \texttt{[l]} determine the position of the numerator.
\begin{LTXexample}[style=myLaTeX, pos=o]
\begin{equation*}
x = x_{0} + \cfrac{y_{0}}{
x_{1} + \cfrac{y_{1}}{
x_{2} + \cfrac[l]{y_{2}}{
x_{3} + \cfrac[r]{y_{3}}{
x_{4} + \cdots}}}}
\end{equation*}
\end{LTXexample} 

% Combinatorics %
\subsection{Combinatorics\index{combinatorics}}
\begin{table}[H]
\centering
\small
\begin{tabular}{|ccc|}
\hline
\textbf{Description}                             & \textbf{Command}                           & \textbf{Output}    \\                              
Factorial & \texttt{n!} & $n!$ \\ 
Double Factorial & \texttt{n!!} & $n!!$ \\
Derangement & \texttt{!n} & $!n$ \\
Combination & \command{binom}\texttt{\{n\}\{k\}} & $\binom{n}{k}$ \\
Multinomial Coefficient & \command{binom}\texttt{\{n\}\{k\_{}1, k\_{}2, \command{ldots}, k\_{}r\}} & $\binom{n}{k_1, k_2, \ldots, k_r}$ \\
Multiset & \texttt{\command{left}(\command{binom}\texttt{\{n\}\{k\}}\command{right})} & $\left(\binom{n}{k}\right)$ \\
Primorial & \texttt{n\command{\#}} & $n\#$ \\
\hline
\end{tabular}
\end{table}
\index{factorial}
\index{double factorial}
\index{derangement}
\index{combination}
\index{permutation}
\index{multinomial coefficient}
\index{multiset}
\index{primorial}
$\\$
You can also use \command{dbinom} for a \textbf{d}isplay mode sized binomial and \command{tbinom} for a \textbf{t}ext mode sized binomial.

% Probability & Statistics % 
\section{Stochastics (Probability \& Statistics) \index{probability}\index{statistics}\index{stochastics}}

% Probability %
\subsection{Probability\index{probability}}
\begin{table}[H]
\centering
\small
\begin{tabular}{|ccc|}
\hline
\textbf{Description}                             & \textbf{Command}                           & \textbf{Output}    \\                              
Probability Measure & \texttt{P(E)} & $P(E)$ \\ 
Conditional Probability & \texttt{P(A \command{mid} B)} & $P(A \mid B)$ \\ 
Expected Value & \texttt{E(X)} & $E(X)$ \\
Variance & \texttt{\textbackslash mathrm\{Var\}(X)} & $\mathrm{Var}(X)$ \\
Standard Deviation & \command{sigma}\texttt{(X)} & $\sigma(X)$ \\
Covariance & \texttt{\textbackslash mathrm\{Cov\}(X,  Y)} & $\mathrm{Cov}(X)$ \\
Correlation & \command{rho}\texttt{(X, Y)} & $\rho(X, Y)$ \\
Probability Distribution & \texttt{X \command{sim} Y} & $X \sim Y$ \\
\hline
\end{tabular}
\end{table}
\index{probability}
\index{conditional probability}
\index{expected value}
\index{standard deviation}
\index{variance}
\index{correlation}
\index{probability distribution}

% Statistics %
\subsection{Statistics\index{statistics}}
\begin{table}[H]
\centering
\small
\begin{tabular}{|ccc|}
\hline
\textbf{Description}                             & \textbf{Command}                           & \textbf{Output}    \\                              
Mean & \texttt{\command{overline}\{x\}} & $\overline{x}$ \\ 
Estimator &  \command{hat}\texttt{\{p\}}& $\hat{p}$ \\ 
\hline
\end{tabular}
\end{table}
\index{mean}
\index{estimator}

% Linear Algebra %
\section{Linear Algebra\index{linear algebra}}

% Vectors %
\subsection{Vectors\index{vectors}}
\label{vectors}
Vectors are denoted using \command{vec}.
\begin{LTXexample}[style=myLaTeX, pos=o]
$\vec{a}$
\end{LTXexample} 
$\\$
Bold vectors require \command{boldsymbol}. Typing this out can be cumbersome, so define a new command in the preamble.
\begin{lstlisting}[style=myLaTeX, columns=fullflexible]
\newcommand{\bvec}[1]{\boldsymbol{#1}}
\end{lstlisting} 
$\\$ 
Using the new command:
\begin{LTXexample}[style=myLaTeX, pos=o]
$\bvec{a}$
\end{LTXexample} 
$\\$
Vectors are defined within a \texttt{matrix}\footnote{The \texttt{array} environment does the same thing but is not preferred.}, \texttt{pmatrix}, \texttt{bmatrix}, or \texttt{Bmatrix} environment. \index{environments!matrix@\texttt{matrix}}\index{environments!pmatrix@\texttt{pmatrix}}\index{environments!bmatrix@\texttt{bmatrix}}\index{environments!Bmatrix@\texttt{Bmatrix}} \\\\
Row vectors:
\begin{LTXexample}[style=myLaTeX, pos=o]
% Row Vector (no fences)
\begin{equation*}
\begin{matrix}
1 & 2 & 3 
\end{matrix}
\end{equation*}
\end{LTXexample} 

\begin{LTXexample}[style=myLaTeX, pos=o]
% Row Vector (round brackets)
\begin{equation*}
\begin{pmatrix}
1 & 2 & 3 
\end{pmatrix}
\end{equation*}
\end{LTXexample} 

\begin{LTXexample}[style=myLaTeX, pos=o]
% Row Vector (square brackets)
\begin{equation*}
\begin{bmatrix}
1 & 2 & 3 
\end{bmatrix}
\end{equation*}
\end{LTXexample} 

\begin{LTXexample}[style=myLaTeX, pos=o]
% Row Vector (curly braces)
\begin{equation*}
\begin{Bmatrix}
1 & 2 & 3 
\end{Bmatrix}
\end{equation*}
\end{LTXexample} 
\index{row vectors}
$\\$
Column vectors:
\begin{LTXexample}[style=myLaTeX, pos=o]
% Column Vector (no delimiters)
\begin{equation*}
\begin{matrix}
1 \\
2 \\
\vdots \\
3
\end{matrix}
\end{equation*}
\end{LTXexample} 

\begin{LTXexample}[style=myLaTeX, pos=o]
% Column Vector (round brackets)
\begin{equation*}
\begin{pmatrix}
1 \\
2 \\
\vdots \\
3
\end{pmatrix}
\end{equation*}
\end{LTXexample} 

\begin{LTXexample}[style=myLaTeX, pos=o]
% Column Vector (square brackets)
\begin{equation*}
\begin{bmatrix}
1 \\
2 \\
\vdots \\
3
\end{bmatrix}
\end{equation*}
\end{LTXexample} 

\begin{LTXexample}[style=myLaTeX, pos=o]
% Column Vector (curly braces)
\begin{equation*}
\begin{Bmatrix}
1 \\
2 \\
\vdots \\
3
\end{Bmatrix}
\end{equation*}
\end{LTXexample} 
\index{column vectors}

% Matrices %
\subsection{Matrices\index{matrices}}
Use the exact same environments mentioned in \ref{vectors}.
\begin{LTXexample}[style=myLaTeX, pos=o]
% Matrix (no delimiters)
\begin{equation*}
\begin{matrix}
1 & 2 & 3 \\
4 & 5 & 6
\end{matrix}
\end{equation*}
\end{LTXexample} 

\begin{LTXexample}[style=myLaTeX, pos=o]
% Matrix (round brackets)
\begin{equation*}
\begin{pmatrix}
a_{11} & \cdots & a_{1n} \\
\vdots & \ddots & \vdots \\
a_{m1} & \cdots & a_{mn} 
\end{pmatrix}
\end{equation*}
\end{LTXexample} 

\begin{LTXexample}[style=myLaTeX, pos=o]
% Matrix (square brackets)
\begin{equation*}
\begin{bmatrix}
1 & 2 \\
4 & 5
\end{bmatrix}
\end{equation*}
\end{LTXexample} 

\begin{LTXexample}[style=myLaTeX, pos=o]
% Matrix (curly braces)
\begin{equation*}
\begin{Bmatrix}
1 & 2 \\
3 & 4 \\
5 & 6
\end{Bmatrix}
\end{equation*}
\end{LTXexample} 
$\\$
If you need matrices with different delimiters, then you add them to a plain \texttt{matrix} using \command{left} and \command{right}.
\begin{LTXexample}[style=myLaTeX, pos=o]
% Matrix (custom delimiters)
$
\left(
\begin{matrix}
1 & 2 \\
3 & 4 
\end{matrix}
\right]
$, 
$
\left\lceil
\begin{matrix}
1 & 2 \\
3 & 4
\end{matrix}
\right\rfloor
$
\end{LTXexample} 
$\\$
Even in inline mode, matrices are in display style. For smaller matrices, use \texttt{smallmatrix},  \texttt{psmallmatrix}, or  \texttt{bsmallmatrix}.\index{environments!smallmatrix@\texttt{smallmatrix}}\index{environments!psmallmatrix@\texttt{psmallmatrix}}\index{environments!bsmallmatrix@\texttt{bsmallmatrix}}
\begin{LTXexample}[style=myLaTeX, pos=o]
% Small Matrix (no delimiters)
$
\begin{smallmatrix}
1 & 2 \\
3 & 4 
\end{smallmatrix}
$ is a $2 \times 2$ matrix.
\end{LTXexample} 

\begin{LTXexample}[style=myLaTeX, pos=o]
% Small Matrix (round brackets)
$
\begin{psmallmatrix}
1 & 2 \\
3 & 4 
\end{psmallmatrix}
$ is a $2 \times 2$ matrix.
\end{LTXexample} 

\begin{LTXexample}[style=myLaTeX, pos=o]
% Small Matrix (square brackets)
$
\begin{bsmallmatrix}
1 & 2 \\
3 & 4 
\end{bsmallmatrix}
$ is a $2 \times 2$ matrix.
\end{LTXexample}

\begin{LTXexample}[style=myLaTeX, pos=o]
% Small Matrix (custom brackets)
$
\left(
\begin{smallmatrix}
1 & 2 \\
3 & 4 
\end{smallmatrix}
\right\}
$ is a $2 \times 2$ matrix.
\end{LTXexample}  

% Determinants %
\subsection{Determinants\index{determinants}}
Use the \texttt{vmatrix} environment.\index{environments!@vmatrix\texttt{vmatrix}}
\begin{LTXexample}[style=myLaTeX, pos=o]
\begin{equation*}
\begin{vmatrix}
1 & 2 \\
3 & 4 
\end{vmatrix}
= 1 \cdot 4 - 2 \cdot 3 = -2
\end{equation*}
\end{LTXexample}  
$\\$
An alternative is:
\begin{LTXexample}[style=myLaTeX, pos=o]
\begin{equation*}
\left|
\begin{matrix}
1 & 2 \\
3 & 4 
\end{matrix}
\right|
= 1 \cdot 4 - 2 \cdot 3 = -2
\end{equation*}
\end{LTXexample}  
$\\$ 
You can also use \command{det}.
\begin{LTXexample}[style=myLaTeX, pos=o]
The determinant of $A$ is
\begin{equation*}
\det{\left(
\begin{pmatrix}
1 & 2 \\
3 & 4 
\end{pmatrix}
\right)}
= 1 \cdot 4 - 2 \cdot 3 = -2
\end{equation*}
\end{LTXexample}  

% Matrix Norm %
\subsection{Matrix Norm\index{matrix norm}\index{norm}}
Use the \texttt{Vmatrix} environment.\index{environments!Vmatrix@\texttt{Vmatrix}}
\begin{LTXexample}[style=myLaTeX, pos=o]
\begin{equation*}
\begin{Vmatrix}
1 & 2 \\
3 & 4 
\end{Vmatrix}
\end{equation*}
\end{LTXexample}  
$\\$
An alternative is:
\begin{LTXexample}[style=myLaTeX, pos=o]
\begin{equation*}
\left\|
\begin{matrix}
1 & 2 \\
3 & 4 
\end{matrix}
\right\|
\end{equation*}
\end{LTXexample}  

% Vector Calculus %
\subsection{Vector Calculus\index{vector calculus}}
\begin{table}[H]
\centering
\small
\begin{tabular}{|ccc|}
\hline
\textbf{Description}                             & \textbf{Command}                           & \textbf{Output}    \\                              
Dot Product & \texttt{v \command{cdot} w} & $v \cdot w$ \\ 
Inner Product & \texttt{\command{langle} v, w \command{rangle}} & $\langle v, w \rangle$ \\ 
Cross Product & \texttt{v \command{times} w} & $v \times w$ \\ 
Triple Product & \texttt{(u, v, w)} & $(u, v, w)$ \\ 
Dyadic Product & \texttt{v \command{otimes} w} & $v \otimes w$ \\ 
Unit Vector & \command{hat}\texttt{\{v\}}& $\hat{v}$ \\ 
\hline
\end{tabular}
\end{table}
\index{dot product}
\index{inner product}
\index{cross product}
\index{triple product}
\index{unit vector}
\index{dyadic product}

% Matrix Operations %
\subsection{Matrix Operations\index{matrix operations}}
\begin{table}[H]
\centering
\small
\begin{tabular}{|ccc|}
\hline
\textbf{Description}                             & \textbf{Command}                           & \textbf{Output}    \\                              
Matrix Multiplication & \texttt{A \command{cdot} B} & $A \cdot B$ \\ 
Hadamard Product & \texttt{A \command{circ} B} & $A \circ B$ \\ 
Kronecker Product & \texttt{A \command{otimes} B} & $A \otimes B$ \\ 
Matrix Transpose & \texttt{A\^{}\{T\}} & $A^{T}$ \\ 
Conjugate Transpose & \texttt{A\^{}\{*\}} & $A^{*}$ \\ 
Inverse Matrix & \texttt{A\^{}\{-1\}} & $A^{-1}$ \\ 
Trace & \texttt{\command{mathrm}\{tr\}(A)} & $\mathrm{tr}(A)$ \\ 
Determinant & \texttt{\command{det}(A)} & $\det(A)$ \\ 
Determinant & \texttt{|A|} & $|A|$ \\ 
Matrix Norm & \texttt{\textbackslash|A\textbackslash|} & $\|A\|$ \\ 
Rank & \texttt{\command{mathrm}\{rank\}(A)} & $\mathrm{rank}(A)$ \\ 
Span & \texttt{\command{mathrm}\{span\}(A)} & $\mathrm{span}(A)$ \\ 
\hline
\end{tabular}
\end{table}
\index{matrix multiplication}\index{multiplication}
\index{Hadamard product}
\index{Kronecker product}
\index{transpose}\index{matrix multiplication}
\index{conjugate transpose}
\index{matrix inverse}\index{inverse matrix}
\index{determinant}
\index{trace}
\index{matrix norm}

% Vector Spaces %
\subsection{Vector Spaces \index{vector spaces}}
\begin{table}[H]
\centering
\small
\begin{tabular}{|ccc|}
\hline
\textbf{Description}                             & \textbf{Command}                           & \textbf{Output}    \\                              
Kernel & \texttt{\command{ker}\{W\}} & $\ker{W}$ \\ 
Dimension & \texttt{\command{dim}\{W\}} & $\dim{W}$ \\ 
Degree & \texttt{\command{degree}\{P(x)\}} & $\deg{P(x)}$ \\ 
Direct Sum & \texttt{V \command{oplus} W} & $V \oplus W$ \\ 
Direct Product & \texttt{V \command{times} W} & $V \times W$ \\ 
Tensor Product & \texttt{V \command{otimes} W} & $V \otimes W$ \\ 
Quotient Space & \texttt{V / W} & $V / W$ \\ 
Orthogonal Complement & \texttt{W\^{}\{\command{perp}\}} & $A^{\perp}$ \\ 
Dual Space & \texttt{V\^{}\{*\}} & $V^{*}$ \\
Linear Hull & \texttt{\command{langle} X \command{rangle}} & $\langle X \rangle$ \\ \hline
\end{tabular}
\end{table}
\index{kernel}
\index{dimension}
\index{direct sum}
\index{direct product}
\index{tensor product}
\index{quotient space}
\index{orthogonal complement }
\index{dual space}
\index{linear hull}

% Overriding Math Styles %
\section{Overriding Default Math Styles}
Suppose you want a display mode style sum in between text. How do you do that? Fortunately, \LaTeX{} provides commands to override the default style that math is typeset.
 \begin{itemize}
\item \command{textstyle} - inline math style.
\item \command{displaystyle} - display math style.
\item \command{scriptstyle} - sub/superscript math style.
\item \command{scriptscriptstyle} - second order sub/superscript math style.
\end{itemize}
$\\$ 
These commands are useful with sums, products, integral, and limits.
\begin{LTXexample}[style=myLaTeX, pos=o]
% Display Mode
\[\sum_{n=1}^{10}{n}\] 

% Text Style in Display Mode
\[\textstyle\sum_{n=1}^{10}{n}\] 

% Other Styles
$\scriptscriptstyle\sum_{n=1}^{10}
{n}$, 
$\scriptstyle\sum_{n=1}^{10}{n}$,  
$\sum_{n=1}^{10}{n}$, 
$\displaystyle\sum_{n=1}^{10}{n}$ 
\end{LTXexample} 

% Coloring Math %
\section{Coloring Math \index{color}}
Coloring math is similar to coloring text (refer to \ref{colors} on page \pageref{colors}).
\begin{LTXexample}[style=myLaTeX, pos=o]
\[\frac{\textcolor{blue}{5}}{10}= \frac{1}{\textcolor{red}{10}}\]
\end{LTXexample}

% Math HW %
\section{Homework \index{homework}}
There are a few templates for homework assignments that I have uploaded to \href{https://github.com/Prabhav10/LaTeX}{GitHub}. More templates can be found on \href{https://www.overleaf.com/gallery/tagged/homework}{Overleaf}.

% Math LaTeX Resources %
\subsubsection{Helpful Resources\index{resources}}
\begin{enumerate}
\item \href{https://en.wikibooks.org/wiki/LaTeX/Mathematics}{Wikibooks} - a thorough guide for typesetting mathematics.
\item \href{http://tug.ctan.org/info/short-math-guide/short-math-guide.pdf}{AMS Math Guide for \LaTeX} - a guide to \LaTeX{} by the American Mathematical Society.
\item \href{https://faculty.math.illinois.edu/~west/grammar.html}{The Grammar of Mathematics} - how to write math.
\end{enumerate}

% CHAPTER 7 %
\chapter{Structures}

% Lists %
\section{Lists\index{lists}}
Different environments render different  lists.
\begin{itemize}
\item \texttt{itemize} - unordered list (bullet points).\index{environments!itemize@\texttt{itemize}}
\item \texttt{enumerate} - ordered list (numbers).\index{environments!enumerate@\texttt{enumerate}}
\item \texttt{description} - description list (words).\index{environments!description@\texttt{description}}
\end{itemize}
\begin{LTXexample}[style=myLaTeX, pos=o]
Grocery list:
\begin{itemize}
  \item Pineapples
  \item More Pineapples
  \item Even More Pineapples
\end{itemize}
\end{LTXexample} 

\begin{LTXexample}[style=myLaTeX, pos=o]
Premier League Top 4:
\begin{enumerate}
  \item Manchester United
  \item Manchester City
  \item Liverpool
  \item Chelsea
\end{enumerate}
\end{LTXexample} 

\begin{LTXexample}[style=myLaTeX, pos=o]
Bull's Starting Line-up:
\begin{description}
  \item[PG] Lonzo Ball
  \item[SG] Zach Lavine
  \item[SF] DeMar DeRozan
  \item[PF] Javonte Green
  \item[C] Nikola Vučević
\end{description}
\end{LTXexample} 
$\\$
You can also nest lists.
\begin{LTXexample}[style=myLaTeX, pos=o]
Bull's Starting Line-up:
\begin{description}
  \item[PG] Lonzo Ball
    \begin{description}
      \item Bench
        \begin{itemize}
          \item[\textbf{\#6}] Alex Caruso
          \item Coby White
        \end{itemize}
      \end{description}
  \item[SG] Zach Lavine
  \item[SF] DeMar DeRozan
  \item[PF] Javonte Green
  \item[C] Nikola Vučević
\end{description}
\end{LTXexample} 
$\\$
More information on lists can be found \href{https://www.overleaf.com/learn/latex/Lists}{here}.

% Tables %
\section{Tables\index{Tables}}
The \texttt{table} and \texttt{tabular} environments are used to create tables\index{environments!tabular@\texttt{tabular}}\index{environments!table@\texttt{table}}.

% Table Environment %
\subsection{The \texttt{table} environment}
\begin{LTXexample}[style=myLaTeX, pos=o]
\begin{table}[c] % t = top of the page; c = center of the page b = bottom of the page

% Title of the table
\caption{Basic Table} 

% Centers table (table is aligned to left by default)
\centering 
\end{table}
\end{LTXexample} 

\begin{itemize}
\item \texttt{c} specifies the position of the table within the page.
\item To place table at precisely the location in the \LaTeX{} code, load the \texttt{float} package and use \texttt{H} instead of \texttt{c}.
\item To right align the table, place the code in a \texttt{flushright} environment.
\end{itemize}
$\\$
To fill in the table contents, start a \texttt{tabular} environment.
\begin{LTXexample}[style=myLaTeX, pos=o]
\begin{center} % You can also use the center environment 
\begin{table}[c] 
\caption{Basic Table} 

% 3 columns: l = left justified contents; c = centered column contents; r = right justified contents
\begin{tabular}{l c r}
1 & 2 & 3 \\
4 & 5 & 6 \\
\end{tabular}
\end{table}
\end{center}
\end{LTXexample} 

\begin{itemize}
\item Column widths and spacing are automatically defined.
\item \texttt{\&} separates columns.
\item \command{\textbackslash} separates rows.
\item For simple tables, you may only need the \texttt{tabular} environment.
\end{itemize}
$\\$
Notice the small gap between the table contents and title. The \texttt{caption} package solves this. Add the following to the preamble.
\begin{lstlisting}[style=myLaTeX, columns=fullflexible]
\usepackage{caption} 
\captionsetup[table]{skip=10pt}
\end{lstlisting} 
$\\$
The \texttt{caption} package provides more customization options. Read \href{http://www.peteryu.ca/tutorials/publishing/latex_captions}{this} tutorial for more information.

% Tabular Environment %
\subsection{The \texttt{tablular} environment}
The \texttt{tablular} environment was introduced in the last section. Let's continue adding features to it.
\begin{LTXexample}[style=myLaTeX, pos=o]
\centering

\begin{tabular}[c]{|| l | c | r ||}
\hline 
Col 1 & Col 2 & Col 3 \\ [0.2ex] % Headings 
1 & 2 & 3 \\
\hline
4 & 5 & 6 \\
\hline
7 & 8 & 9 \\
\hline\hline
\end{tabular}
\end{LTXexample} 

\begin{itemize}
\item \texttt{|} adds a vertical line between columns.
\item \texttt{||} adds a double vertical line between columns.
\item \command{hline} adds a horizontal line between rows. 
\item \command{hline}\command{hline} adds a double horizontal line between rows. 
\item There is no need \command{\textbackslash} after \command{hline}.
\item Add space between rows with square brackets \texttt{[]}
\end{itemize}
\begin{LTXexample}[style=myLaTeX, pos=o]
\centering

\begin{tabular}[c]{l c r}
\hline 
Col 1 & Col 2 & Col 3 \\ [0.2ex]
\cline{2-2}
1 & 2 & 3 \\
\cline{2-3}
4 & 5 & 6 \\
\hline
7 & 8 & 9 \\
\hline
\end{tabular}
\end{LTXexample} 

\begin{itemize}
\item \command{cline}\texttt{\{m-n\}} adds a horizontal line between columns \texttt{m} and \texttt{n}.
\end{itemize}

\begin{LTXexample}[style=myLaTeX, pos=o]
\centering

\begin{tabular}[c]{| l | c | r | p{2.5cm} |}
\hline 
Col 1 & Col 2 & Col 3 & Text \\ [0.2ex] 
1 & 2 & 3 & Numbers from 1-3. \\
\hline
4 & 5 & 6 & Numbers from 4-6. \\
\hline
7 & 8 & 9 & Numbers from 7-9. \\
\hline
\end{tabular}
\end{LTXexample} 

\begin{itemize}
\item \command{p}\texttt{\{2.5cm\}} specifies a paragraph column with text vertically aligned at the top.
\end{itemize}
$\\$
More complex tables involving merging rows and columns. Use \command{multicolumn} to merge cells over multiple columns.
\begin{LTXexample}[style=myLaTeX, pos=o]
\centering

\begin{tabular}[c]{| l | c | r |}
\hline 
Col 1 & Col 2 & Col 3 \\ [0.2ex]
\hline
1 & \multicolumn{2}{c}{2, 3} \\
\hline
\multicolumn{3}{c}{4, 5, 6} \\
\hline
\multicolumn{1}{l}{7} & \multicolumn{1}{c}{8} & \multicolumn{1}{r}{9} \\
\hline
\end{tabular}
\end{LTXexample} 

\begin{itemize}
\item \command{multicolumn} removes the vertical lines, so specify them.
\end{itemize}
\begin{LTXexample}[style=myLaTeX, pos=o]
\centering

\begin{tabular}[c]{| l | c | r |}
\hline 
Col 1 & Col 2 & Col 3 \\ [0.2ex]
\hline
1 & \multicolumn{2}{| c |}{2, 3} \\
\hline
\multicolumn{3}{| c |}{4, 5, 6} \\
\hline
\multicolumn{1}{| l |}{7} & \multicolumn{1}{| c |}{8} & \multicolumn{1}{| r |}{9} \\
\hline
\end{tabular}
\end{LTXexample} 
$\\$
Load the \texttt{multirow} package and use \command{multirow} to merge cells over multiple rows.
\begin{LTXexample}[style=myLaTeX, pos=o]
\centering

\begin{tabular}{| c | c | c |}
\hline
\multicolumn{3}{| c |}{Bulls Roster} \\
\hline
\multirow{2}{*}{Point Guards} 
& PG1 & Lonzo B. \\
& PG2 & Alex C. \\
\hline
\multirow{2}{*}{Shooting Guards} 
& SG1 & Zach L. \\
& SG2 & Ayo D. \\
\hline
\multirow{2}{*}{Small Forwards} 
& SF1 & DeMar D. \\
& SF2 & Derrick J. \\
\hline
\multirow{2}{*}{Power Forwards} 
& PF1 & Patrick W. \\
& PF2 & Javonte G. \\
\hline
\multirow{2}{*}{Centers} 
& C1 & Nikola V. \\
& C2 & Tony B. \\
\hline
\end{tabular}
\end{LTXexample} 

\begin{itemize}
\item \texttt{*} tells \LaTeX{} that the column width is determined by its content.
\end{itemize}

% Table LaTeX Resources %
\subsubsection{Helpful Resources\index{resources}}
\begin{enumerate}
\item \href{https://www.tablesgenerator.com}{Table to \LaTeX{} generators} - converts drawn table to \LaTeX.
\item \href{https://www.overleaf.com/learn/latex/Positioning_images_and_tables#Basic_positioning_2}{Overleaf} - positioning tables.
\item \href{https://en.wikibooks.org/wiki/LaTeX/Tables}{Wikibooks} - an advanced guide for tables.
\end{enumerate}
\index{table spacing}

% Images %
\section{Images\index{graphics}\index{images}}
\begin{enumerate}
\item Save the image in the folder your document is saved in (as a EPS, JPEG, PDF, or PNG).
\item Load the \texttt{graphicx} package.
\item Use \command{includegraphics}.
\end{enumerate}

\begin{LTXexample}[style=myLaTeX, pos=o]
\begin{center}
\includegraphics[width=5cm, height=3cm, angle=0, scale=1]{ronaldo.jpeg}
\end{center}
\end{LTXexample} 
$\\$
Sometimes images and text do not work well together, so images must be placed in a \texttt{figure} environment\index{environments!figure@\texttt{figure}}. It is similar to the \texttt{table} environment in some ways.
\begin{LTXexample}[style=myLaTeX, pos=o]
\begin{center}
\begin{figure}
\caption{The \textbf{SIU}}
\includegraphics[width=5cm, height=3cm, angle=0, scale=1]{ronaldo.jpeg}
\end{figure}
\end{center}
\end{LTXexample} 
$\\$
More information on inserting images can be found \href{https://www.overleaf.com/learn/latex/Inserting_Images}{here}.

% CHAPTER 8 %
\chapter{Navigation}

% Contents %
\section{Table of Contents\index{table of contents}}
Use \command{tableofcontents} in the body of the document.

% List of... %
\section{List of Tables \& Figures\index{list of figures}\index{list of tables}}
Use \command{listoftables} and \command{listoffigures} in the body of the document.

% Abstract %
\section{Abstract\index{abstract}\index{summary}}
Add the following code to the document body:
\begin{lstlisting}[style=myLaTeX, columns=fullflexible]
\chapter*{Abstract}

% Adding Abstract to Table of Contents 
\addcontentsline{toc}{chapter}{Abstract}  
\end{lstlisting} 
$\\$
An alternative solution is to use the \texttt{abstract} environment.\index{environments!abstract@\texttt{abstract}}
\begin{lstlisting}[style=myLaTeX, columns=fullflexible]
\begin{abstract}
This guide serves as an introduction to \LaTeX{}. I hope new users find it useful.
\end{abstract}
\end{lstlisting} 

% Acknowledgements %
\section{Acknowledgements\index{acknowledgements}\index{thanks}}
Add the following code to the document body:
\begin{lstlisting}[style=myLaTeX, columns=fullflexible]
\chapter*{Acknowledgements}

% Adding Acknowledgements to Table of Contents 
\addcontentsline{toc}{chapter}{Acknowledgements}  
\end{lstlisting} 

% Appendix %
\section{Appendix\index{appendix}}
Load the \texttt{appendix} package as follows:
\begin{lstlisting}[style=myLaTeX, columns=fullflexible]
% Preamble
\usepackage[toc]{appendix} % Includes appendices in Table of Contents

% Body
\begin{appendices}
\chapter{Riemann Hypothesis Proof}
Sir Michael Atiyah claims the proof for the Riemann Hypothesis is as follows...
\end{appendices}
\end{lstlisting} 

% Bibliography %
\section{Bibliography\index{bibliography}\index{references}}
\label{references}
Watch \href{https://www.youtube.com/watch?v=KS9GvK7cvmo}{this} video. A few things to remember:
\begin{itemize}
\item Run the compilers below in the order stated:
\begin{enumerate}
\item \LaTeX{}\footnote{You can use \XeLaTeX{} or \LuaLaTeX{} instead.} 
\item Bib\TeX 
\item \LaTeX{} ($\mathsf{\times 2}$)
\end{enumerate}
\item Other bibliography styles can be found \href{https://www.overleaf.com/learn/latex/Bibtex_bibliography_styles\#Table_of_stylename_values}{here}.
\item When you use \command{bibliography}, the .bib file name must be within the \texttt{\{\}}.
\item When you use \command{cite}, the name must within \texttt{\{\}} must match the name in the .bib file.
\end{itemize}
\index{Bib\TeX}

% Index %
\section{Index\index{index}}
Load the \texttt{imakeidx} package and type the following code:
\begin{lstlisting}[style=myLaTeX, columns=fullflexible]
% Preamble
\usepackage{imakeidx}

% Alphabetical Index 
\begin{filecontents*}{\jobname.mst}
headings_flag 1
heading_prefix   "\\par\\penalty-50\\textbf{"
heading_suffix   "}\\\\\*\~\\\\\*"
symhead_positive "Symbols"
symhead_negative "symbols"
numhead_positive "Numbers"
numhead_negative "numbers"
delim_0 ",\~"
\end{filecontents*}

% Making the Index 
\makeindex[intoc]

% Body

% Making an Index entry
This is the first index\index{first entry} entry.

% Printing the Index
\printindex
\end{lstlisting} 
$\\$
Next, run the compilers below in the order stated:
\begin{enumerate}
\item \LaTeX{}\footnote{Refer to the footnote in \ref{references}}
\item MakeIndex
\item \LaTeX{} ($\mathsf{\times 2}$)
\end{enumerate}
To change the style of index entries, refer to \href{https://en.wikibooks.org/wiki/LaTeX/Indexing\#Sophisticated_indexing}{this} table.

% Hyperlinks %
\section{Hyperlinks\index{hyperlinks}\index{links}}
Load the \texttt{href} package.
\begin{lstlisting}[style=myLaTeX, columns=fullflexible]
% Preamble
\usepackage[colorlinks, urlcolor=blue]{href}
\end{lstlisting} 
$\\$
Use \command{href} to add a link.
\begin{LTXexample}[style=myLaTeX, pos=o]
This is a \href{https://www.google.com}
{link}.
\end{LTXexample} 
$\\$
If you just want a URL, then use \command{url}.
\begin{LTXexample}[style=myLaTeX, pos=o]
\url{https://www.google.com} provides a pretty good search engine.
\end{LTXexample} 
$\\$
You can also add your email address.
\begin{LTXexample}[style=myLaTeX, pos=o]
\href{mailto:prabhavkumar10@gmail.com}
{Say Hi!}
\end{LTXexample} 
$\\$
More information about hyperlinks can be found \href{https://www.overleaf.com/learn/latex/Hyperlinks}{here}.

% CHAPTER 9 % 
\chapter{Drawing\index{drawing}\index{art}}

TikZ is the most powerful graphics tool in \LaTeX. While it is quite complex, I introduce the basics.

% Lines %
\section{Lines\index{lines}}
Load the \texttt{tikz} package.
\begin{lstlisting}[style=myLaTeX, columns=fullflexible]
% Preamble
\usepackage{tikz}
\end{lstlisting} 
$\\$
Use \command{tikz}\texttt{$\ldots$;} to draw inline. \textbf{\texttt{;} marks the end of the instruction and is necessary}.
\begin{LTXexample}[style=myLaTeX, pos=o]
\tikz \draw (0, 0) - - (1, 0); is a straight line
\end{LTXexample} 
$\\$
Use the \texttt{tikzpicture} environment for larger pictures.\index{environments!tikzpicture@\texttt{tikzpicture}}
\begin{LTXexample}[style=myLaTeX, pos=o]
\centering 
\begin{tikzpicture}

% Drawing Grid Lines
\draw[help lines] (-1, -1) grid (1, 1);

% Drawing a Path
\draw (0, 0) - - (1, 0) -- (1, 1) -- (0, 1) -- (-1, 1) -- (-1, 0) -- (-1, -1);

\end{tikzpicture}

\end{LTXexample}  
$\\$ 
Drawing the same path using \texttt{|}:
\begin{LTXexample}[style=myLaTeX, pos=o]
\centering 
\begin{tikzpicture}
\draw[help lines] (-1, -1) grid (1, 1);

% Drawing Same Path
\draw (0, 0) -| (1, 1) -| (-1, -1);

\end{tikzpicture}
\end{LTXexample}  

% Points %
\section{Points\index{points}}
\begin{LTXexample}[style=myLaTeX, pos=o]
\centering 
\begin{tikzpicture}

% Drawing Grid Lines
\draw[help lines] (-1, -1) grid (1, 1);

% Drawing a Path
\draw (0, 0) -| (1, 1) -| (-1, -1);

% Starting & Ending Points
\filldraw (0,0) circle (2pt); 
\filldraw (-1, -1) circle (2pt); 

\end{tikzpicture}
\end{LTXexample}  

% Curved Lines %
\section{Curved Lines\index{curved lines}}
\begin{LTXexample}[style=myLaTeX, pos=o]
\centering 
\begin{tikzpicture}
\draw[help lines] (-1, -1) grid (1, 1);

% Drawing a Parabola
\draw (0, 0) parabola (1, 1);

% Drawing a Curved Line
\draw (-1, -1) .. controls (-0.5, 0) and (0.5, 0) .. (1, -1);

\end{tikzpicture}
\end{LTXexample}  
\texttt{(-1, -1)} and \texttt{(1, -1)} are the start and end points, respectively. \texttt{(-0.5, 0)} and \texttt{(0.5, 0)} act like magnets. Make sure there is no whitespace between the 2 periods before and after \texttt{controls}.

% Shapes %
\section{Shapes\index{shapes}}

% Circle %
A circle centered at the origin of radius 1\index{circle}:
\begin{LTXexample}[style=myLaTeX, pos=o]
\centering 
\begin{tikzpicture}
\draw (0, 0) circle (1);
\end{tikzpicture}
\end{LTXexample}  
$\\$
% Triangle %
A triangle\index{triangle}:
\begin{LTXexample}[style=myLaTeX, pos=o]
\centering 
\begin{tikzpicture}
\draw (0, 0) -- (1, 0) -- (1, 1) -- cycle;
\end{tikzpicture}
\end{LTXexample}  
$\\$
% Rectangle %
A rectangle\index{rectangle}:
\begin{LTXexample}[style=myLaTeX, pos=o]
\centering 
\begin{tikzpicture}
\draw (0, 0) rectangle (1, 2);
\end{tikzpicture}
\end{LTXexample}  
$\\$
% Square %
A square\index{square}:
\begin{LTXexample}[style=myLaTeX, pos=o]
\centering 
\begin{tikzpicture}
\draw (0, 0) rectangle (2, 2);
\end{tikzpicture}
\end{LTXexample}  
$\\$
% Ellipse %
An ellipse centered at the origin with $x$ and $y$-direction radii of 1 and 0.5\index{ellipse}:
\begin{LTXexample}[style=myLaTeX, pos=o]
\centering 
\begin{tikzpicture}
\draw (0, 0) ellipse (1 and 0.5);
\end{tikzpicture}
\end{LTXexample}  
$\\$
% Arc %
An arc of radius 1 from 0 to 90 degrees\index{arc}:
\begin{LTXexample}[style=myLaTeX, pos=o]
\centering 
\begin{tikzpicture}
\draw (0, 0) arc (0:90:1);
\end{tikzpicture}
\end{LTXexample}  

% Scaling %
\section{Scaling\index{scaling}}
Scaling a drawing by a factor of 2:
\begin{LTXexample}[style=myLaTeX, pos=o]
\centering 
\begin{tikzpicture}[scale=2]
\draw[help lines] (-1, -1) grid (1, 1);
\draw (0, 0) circle (1);
\end{tikzpicture}
\end{LTXexample}  
$\\$
Scaling across the $x$-dimension:
\begin{LTXexample}[style=myLaTeX, pos=o]
\centering 
\begin{tikzpicture}[xscale=2]
\draw[help lines] (-1, -1) grid (1, 1);
\draw (0, 0) circle (1);
\end{tikzpicture}
\end{LTXexample}  
$\\$
Scaling across the $x$ and $y$-dimensions:
\begin{LTXexample}[style=myLaTeX, pos=o]
\centering 
\begin{tikzpicture}[xscale=0.5, yscale=2]
\draw[help lines] (-1, -1) grid (1, 1);
\draw (0, 0) circle (1);
\end{tikzpicture}
\end{LTXexample}  

% Decorating Lines %
\section{Decorating Lines\index{decorating lines}}

% Arrows %
\subsection{Arrows\index{arrows}}
\begin{LTXexample}[style=myLaTeX, pos=o]
\centering 
\begin{tikzpicture}
\draw [->] (0, 0) -- (0, 1);
\draw [<-] (1, 0) -- (1, 1);
\draw [<-|] (2, 0) -- (2, 1);
\draw [<->] (3, 0) -- (3, 1);
\end{tikzpicture}
\end{LTXexample}  

% Thickness %
\subsection{Line Thickness\index{thickness}}
\begin{LTXexample}[style=myLaTeX, pos=o]
\centering 
\begin{tikzpicture}

% Pre-defined Thickness %
\draw [ultra thick] (0, 0) -- (0, 1);
\draw [thick] (1, 0) -- (1, 1);
\draw [thin] (2, 0) -- (2, 1);
\draw [very thin] (3, 0) -- (3, 1);

% Custom Thickness %
\draw [line width=3pt] (4, 0) -- (4, 1);

\end{tikzpicture}
\end{LTXexample}  

% Line Styles %
\subsection{Line Styles\index{line styles}\index{dashed lines}\index{dotted lines}}
\begin{LTXexample}[style=myLaTeX, pos=o]
\centering 
\begin{tikzpicture}
\draw [dashed, ultra thick] (0, 0) -- (0, 1);
\draw [dashed, thick] (1, 0) -- (1, 1);
\draw [dotted] (2, 0) -- (2, 1);
\end{tikzpicture}
\end{LTXexample}  

% Colors %
\subsection{Line Color\index{color}\index{line color}}
\begin{LTXexample}[style=myLaTeX, pos=o]
\centering 
\begin{tikzpicture}
\draw [red, dashed, ultra thick] (0, 0) -- (0, 1);
\draw [blue, dashed, thick] (1, 0) -- (1, 1);
\draw [magenta] (2, 0) -- (2, 1);
\end{tikzpicture}
\end{LTXexample}  
$\\$
Refer to \ref{colors} to use different colours.

% Grid Lines %
\subsection{Grid Lines\index{grid lines}}
Custom grid lines:
\begin{LTXexample}[style=myLaTeX, pos=o]
\centering 
\begin{tikzpicture}
\draw[step=1, blue, thin] (-1, -1) grid (1, 1);
\end{tikzpicture}
\end{LTXexample}  
$\\$
Removing outer border:
\begin{LTXexample}[style=myLaTeX, pos=o]
\centering 
\begin{tikzpicture}
\draw[step=1, blue, thin] (-1.9, -1.9) grid (1.9, 1.9);
\end{tikzpicture}
\end{LTXexample}  

% Repetition %
\section{Repetition\index{repetition}\index{for loop}}
If you need to reuse lines of code to draw similar things, use \command{foreach}.
\begin{LTXexample}[style=myLaTeX, pos=o]
\centering 
\begin{tikzpicture}

% Vertical or Horizontal Parallel Lines?
\foreach \x in {0,...,100} {
\draw [red, dashed, ultra thin] (\x * 0.015, 0) -- (\x * 0.015, 1);
};

\end{tikzpicture}
\end{LTXexample}  
$\\$
You don't need to enter math mode to do math with \texttt{\textbackslash x}.

% TikZ LaTeX Resources %
\subsubsection{Helpful Resources\index{resources}}
\label{tikzresources}
I have only scratched the surface of TikZ, so please use these resources, especially if you want to create art.
\begin{enumerate}
\item \href{https://ctan.org/pkg/visualtikz}{My Favorite TikZ Manual} - learn TikZ visually.
\item \href{http://cremeronline.com/LaTeX/minimaltikz.pdf}{Minimal Introduction TiKZ} - a very minimal introduction to TikZ.
\item \href{https://ctan.org/pkg/pgf?lang=en}{Another TikZ Manual} - a comprehensive guide for TikZ.
\item \href{http://www.texample.net/tikz/}{Examples} - learn TikZ through examples.
\item \href{https://tikz.net}{STEM-related Drawings} - STEM-related TikZ drawings and their code.
\end{enumerate}

% CHAPTER 10 % 

% LaTeX with other subjects %
\chapter{Extending \LaTeX{}}

% Physics % 
\section{Physics\index{physics}}
Physics has a lot of diagrams, so TikZ is important. The rest is basically math\footnote{Open to debate.}. Refer to resource 5 in \ref{tikzresources}.   

% Circuits %
\subsection{Circuits\index{circuits}\index{electricity}}
My favorite physics-related package is \texttt{circuitikz}.

\begin{LTXexample}[style=myLaTeX, pos=o]
\centering 
\begin{circuitikz}
\draw (0,0) to [lamp] (4,0); 
\draw (4,0) to (4, -2); 
\draw (4, -2) to [battery] (0, -2); 
\draw (0, -2) to [fuse] (0, 0); 
\end{circuitikz}
\end{LTXexample}  
$\\$
More information can be found \href{https://www.overleaf.com/learn/latex/CircuiTikz_package}{here}.

% Chemistry %
\section{Chemistry\index{chemistry}}

% Basics %
\subsection{Basics}
Load the \texttt{mhchem} package and use \command{ce} in math mode to write formulae.
\begin{LTXexample}[style=myLaTeX, pos=o]
$\ce{H2O}$ \\
$\ce{Cl-}$ \\
$\ce{CrO4^{2}-}$ \\
$\ce{(C3H4O2)2}$ 
\end{LTXexample}  
$\\$
Add the amount before the formula.
\begin{LTXexample}[style=myLaTeX, pos=o]
$\ce{2O2}$ \\
$\ce{3/4Cl2}$ 
\end{LTXexample}  
$\\$
Displaying isotopes:
\begin{LTXexample}[style=myLaTeX, pos=o]
$\ce{^{14}_{6}C}$ \\
$\ce{^{17}_{6}C+}$
\end{LTXexample} 

% Reactions %
\subsection{Reactions\index{chemical reactions}}
\begin{LTXexample}[style=myLaTeX, pos=o]
$\ce{H2 + O -> H2O}$ \\
$\ce{H2 + O ->[\text{heat}] H2O}$ \\
$\ce{H2 + O <- H2O}$ \\
$\ce{CO2 + C <=> 2CO}$
\end{LTXexample} 
$\\$
More information on \texttt{mhchem} can be found \href{https://docs.moodle.org/311/en/Chemistry_notation_using_mhchem}{here}.

% Drawing Chemical Formuale %
\subsection{Drawing Chemical Formulae\index{atoms}\index{molecules}}
Load the \texttt{chemfig} package and use \command{chemfig}.
 
 % Bonds %
\subsubsection{Bonds\index{bonds}}
\begin{LTXexample}[style=myLaTeX, pos=o]
$\chemfig{O - H}$ \\
$\chemfig{O = H}$ \\
$\chemfig{O ~ H}$ \\
$\chemfig{O > H}$ \\
$\chemfig{O >: H}$ \\
$\chemfig{O >| H}$ \\
$\chemfig{O < H}$ \\
$\chemfig{O <: H}$ \\
$\chemfig{O <| H}$ 
\end{LTXexample} 

\subsubsection{Bond Angles}
\begin{LTXexample}[style=myLaTeX, pos=o]
\centering 
Water ($\ce{H2O}$) \vspace{.5cm} \\
\chemfig{H-[1] O-[7] H} \\
\end{LTXexample} 
$\\$
\texttt{[x]} represents $\mathsf{(x * 45)^{\circ}}$. \\\\
A more complex example:
\begin{LTXexample}[style=myLaTeX, pos=o]
\centering 
\chemfig{H-C(-[2] H)(-[6] H)-C(=[1] O)-[7] H}
\end{LTXexample} 

 % Rings %
\subsubsection{Rings\index{rings}}
\begin{LTXexample}[style=myLaTeX, pos=o]
\centering 
\chemfig{C*6(-C=C-C=C-C=)}
\end{LTXexample} 
$\\$
\texttt{C} is the first atom. \texttt{*6} is the number of atoms. \texttt{()} contains the rest of the atoms. \\\\
More information on \texttt{chemfig} can be found \href{https://www.overleaf.com/learn/latex/Chemistry_formulae}{here}.

% Poetry %
\section{Poetry\index{poems}}
Use the \texttt{verse} package.

\begin{LTXexample}[style=myLaTeX, pos=o]
\centering 
\begin{verse}
Roses are Red, \\
Violets are Blue, \\
This guide with help you.
\end{verse}
\end{LTXexample} 
$\\$
More information can be found \href{https://ctan.org/pkg/verse?lang=en}{here}.

% Programming Languages %
\section{Programming Languages\index{computer science}\index{CS}\index{programming languages}}
You may need to type out programming languages in \LaTeX{}.

% Verbatim %
\subsection{\texttt{verbatim} environment\index{environments!verbatim@\texttt{verbatim}}}
The \texttt{verbatim} environment outputs text or code in monospace font.
\begin{LTXexample}[style=myLaTeX, pos=o]
\begin{verbatim}
def add(x, y):
    return x + y
\end{verbatim}
\end{LTXexample} 
$\\$
To type code inline, use \command{verb}.
\begin{LTXexample}[style=myLaTeX, pos=o]
\verb|add()| returns the sum of 2 numbers.
\end{LTXexample} 
$\\$
There should be no space between \command{verb} and \texttt{|}. Any character except a letter or \texttt{*} can be used instead of \texttt{|} as a delimiter. 

% Listings %
\subsection{\texttt{listings} package}
For more customization, use the \texttt{listings} package and the \texttt{lstlisting} environment\index{environments!lstlisting@\texttt{lstlisting}}. For example,
{\footnotesize
\begin{verbatim}
\begin{lstlisting}[language=Python, caption=Python Example]
def add(x, y):
    return x + y
\end{lstlisting}
\end{verbatim}}
$\\$
produces
\begin{lstlisting}[language=Python, caption=Python Example]
def add(x, y):
    if x >= y:
        print(x)
    else:
        print(y)
    return x + y
\end{lstlisting}
$\\$
You can also highlight code, add line numbers, and do many more things. Read \href{https://www.overleaf.com/learn/latex/Code_listing}{this} guide for more information. \\\\
Code highlighting can also be done via the \texttt{minted} package. Read about it \href{https://www.overleaf.com/learn/latex/Code_Highlighting_with_minted}{here}. \textbf{Warning}: \texttt{minted} can cause errors. However, solutions to the most common errors can be found \href{https://tex.stackexchange.com/questions/240913/latex-minted-error}{here}.

\section{PDF Forms\index{forms}\index{pdf forms}}
Use the \texttt{Form} environment\index{environments!Form@\texttt{Form}}.

\begin{LTXexample}[style=myLaTeX, pos=o]
\begin{Form}[action={path/to/submit}]
\begin{tabular}{l}
    \TextField{Name} \\\\
    \CheckBox[width=1em]{Male} \CheckBox[width=1em]{Female} \CheckBox[width=1em]{Other} \\\\
    \Submit{Submit} \quad \Reset{Reset}\\
\end{tabular}
\end{Form}
\end{LTXexample} 
$\\$
More information can be found \href{https://tex.stackexchange.com/questions/14842/creating-fillable-pdfs}{here}. A more thorough example can be found \href{https://www.maltegerken.de/blog/2021/07/creating-pdf-forms-with-latex/}{here}. \textbf{Warning}: making PDF forms in \LaTeX{} can be buggy, so it's probably better to use Adobe Acrobat.

% Emojis %
\section{Emojis\index{emojis}}

% LuaLaTeX %
\subsection{Using \LuaLaTeX}
Use \command{emoji} provided by the \texttt{emoji} package with the \LuaLaTeX compiler as follows: 
\begin{lstlisting}[style=myLaTeX, columns=fullflexible]
% Preamble 
\usepackage{emoji}

% Body
\emoji{flexed-biceps-medium-dark-skin-tone}
\end{lstlisting}
$\\$
A list of emojis provided by the \texttt{emoji} package can be found \href{https://www.ctan.org/pkg/emoji}{here}.

% XeLaTeX %
\subsection{Using \XeLaTeX}
If you need to use the \XeLaTeX compiler, \href{https://www.wfonts.com/font/symbola}{download} the Symbola font on your local PC and do the following:

\begin{LTXexample}[style=myLaTeX, pos=o]
I am {\fontspec{Symbola}\char"1F600}!
\end{LTXexample} 
$\\$
\href{https://unicode.org/emoji/charts/emoji-list.html}{Here} is a list of emoji codes. 

% Emojis as Images %
\subsection{Using Images}
Another option is to insert emojis at images. Read \href{https://kaushikamardas.medium.com/using-emojis-in-latex-c901969efa24}{this} article for more details.

% CV %
\section{Writing a CV\index{resume}\index{CV}}
If you want to write your CV with \LaTeX{}, choose one of the \href{https://www.overleaf.com/gallery/tagged/cv}{templates} and edit accordingly. I have also uploaded a template on \href{https://github.com/Prabhav10/LaTeX}{GitHub}.

% Thesis %
\section{Writing a Thesis\index{thesis}\index{dissertaiton}\index{PhD}}
If you don't know how to write a thesis, read \href{https://www.student.uwa.edu.au/__data/assets/pdf_file/0007/1919239/How-to-write-a-thesis-A-working-guide.pdf}{this} guide. If you want to write your thesis with \LaTeX{}, choose one of the \href{https://www.overleaf.com/latex/templates/tagged/thesis}{templates} and edit accordingly. If you want a video walkthrough, \href{https://www.youtube.com/playlist?list=PLCRFsOKSM7eNGNghvT6QdzsDYwSTZxqjC}{these} are the best videos I have come across.

% Presentations %
\section{Presentations\index{presentations}}
\texttt{beamer} is the document class for presentations. I learnt \texttt{beamer} using Overleaf's \href{https://www.overleaf.com/learn/latex/Beamer}{tutorials}. You can find examples of aesthetically pleasing presentations \href{https://www.overleaf.com/gallery/tagged/presentation}{here}. A list of \texttt{beamer} themes can be found \href{https://github.com/martinbjeldbak/ultimate-beamer-theme-list}{here}.

% CHAPTER 11%
\chapter{Clever Tricks}
Here is a list of \LaTeX{} hacks: \\
\begin{enumerate}
\item \textbf{\command{today}} - prints today's date (\today).
\item \textbf{\command{TeX}} - prints \TeX.
\item \textbf{\command{LaTeX}} - prints \LaTeX.
\item \textbf{Negations} - place \texttt{n} or \command{not} before a math symbol command to get its negation: \command{in} prints $\in$ and \command{not}\command{in} prints $\not\in$ (doesn't always work).
\item \textbf{One Bracket} - use \command{left.} (or \command{right.}) if you only need 1 delimiter:  \verb|\left.\frac{1}{2}\right)| yields $\left.\frac{1}{2}\right)$.
\end{enumerate}

% CHAPTER 12%
\chapter{Common Errors}
\begin{enumerate}
\item \textbf{Too few braces} - \texttt{\textbackslash section\{I am missing a closing brace!}
\item \textbf{Too many braces} - \texttt{\textbackslash section\{I have an extra brace\}\}}
\item \textbf{Non-matching braces} - \texttt{\textbackslash section[My braces don't match\}}
\item \textbf{Missing environment end} - \texttt{\textbackslash begin\{enumerate\} \command{item} Don't forget to add \textbackslash end\{enumerate\} }
\item \textbf{\texttt{hbox} errors} - read \href{https://stuff.mit.edu/afs/sipb/project/www/latex/guide/node28.html}{this}.
\item \textbf{Forgetting to use \texttt{\textbackslash} to escape} - \texttt{\$} does not print \$.
\item \textbf{Forgetting to use math mode} - \verb|a^2 + b^2 = c^2| will cause an error, but \verb|$a^2 + b^2 = c^2$| will not (remember to place mathematical symbols, expressions, and statements in math mode).
\item \textbf{\command{\textbackslash} error} - if you get a \say{There's no line here to end} error, try \verb|$\\$|.
\item \textbf{URL error} - if you can't open a URL, try adding \verb|http://| or \verb|https://|
\item \textbf{Compiler error} - If you use external packages and get an error, you may be using the wrong compiler: e.g. \texttt{fontspec} needs \XeLaTeX{} or \LuaLaTeX.
\item \textbf{Footnote / Index / References / Labels / Links not showing} - recompile the document multiple times to typeset successfully.
\item \textbf{Declaring packages in wrong order} - declare the \texttt{hyperref} package last (as it causes most of the issues). A more comprehensive list of package conflicts can be found \href{http://www.macfreek.nl/memory/LaTeX_package_conflicts}{here}.
\end{enumerate}

% CHAPTER 13 %
\chapter{More Resources}
As \LaTeX{} is open-source, resources are infinite. Popular resources include: \\

\begin{enumerate}
\item \textbf{Big Resources}
\begin{itemize}
\item \href{https://www.google.com}{Search Engine} - \textcolor{blue}{G}\textcolor{red}{o}\textcolor{yellow}{o}\textcolor{blue}{g}\textcolor{green}{l}\textcolor{red}{e}.
\item \href{https://ctan.org}{CTAN} - \LaTeX's humble abode.
\item \href{https://tex.stackexchange.com}{Stack Exchange} - ask questions.
\item \href{https://latex.org/forum/index.php}{\LaTeX{} Forum} - ask questions.
\item \href{https://www.reddit.com/r/LaTeX/}{\LaTeX{} Subreddit} - for reddit fans.
\end{itemize}
\item \textbf{Learn \LaTeX}
\begin{itemize}
\item \href{https://www.overleaf.com/learn}{Overleaf} - learn and write \LaTeX{} online (\underline{highly recommended}).
\item \href{https://en.wikibooks.org/wiki/LaTeX}{Wikibooks} - a more comprehensive \LaTeX{} online guide (\underline{highly recommended}).
\item \href{https://www.youtube.com/playlist?list=PL1D4EAB31D3EBC449}{\LaTeX{} Playlist} - learn \LaTeX{} on YouTube.
\item \href{https://www.youtube.com/channel/UC9rTsvTxJnx1DNrDA3Rqa6A}{Dr Trefor Bazett} - learn \LaTeX{} from a mathematician.
\item \href{http://static.latexstudio.net/wp-content/uploads/2014/09/The+art+of+latex.pdf}{The Art of \LaTeX} - book to learn \LaTeX{}.
\item \href{https://tobi.oetiker.ch/lshort/lshort.pdf}{The Not So Short Introduction to \LaTeX} - Bible of \LaTeX{}.
\item \href{https://www.overleaf.com/gallery}{\LaTeX{}  Gallery} - \LaTeX{} templates (\underline{highly recommended}).
\end{itemize}
\item \textbf{Cheat Sheets}
\begin{itemize}
\item \href{https://wch.github.io/latexsheet/}{\LaTeX{} Cheat Sheet} - 2-page cheat sheet.
\item \href{https://joshua.smcvt.edu/undergradmath/undergradmath.pdf}{\LaTeX{} Math Cheat Sheet} - Math cheat sheet. 
\item \href{https://users.dickinson.edu/~richesod/latex/latexcheatsheet.pdf}{\LaTeX{} Quick Guide} - 2-page guide.
\end{itemize}
\item \textbf{Some Pretty Cool Stuff}
\begin{itemize}
\item \href{https://mathpix.com}{Mathpix Snip Notes} - convert images and pdf documents to \LaTeX{}.
\item \href{https://castel.dev}{\LaTeX{} + Vim} - writing \LaTeX{} in Vim.
\item \href{https://www.notion.so/help/math-equations}{\LaTeX{} + Notion} - writing \LaTeX{} in Notion.
\item \href{https://bitbucket.org/kleberj/logicpuzzle/wiki/Home}{\texttt{logicpuzzle}} - create puzzles (sudoku, battleship etc.) with \LaTeX.
\item \href{https://www.overleaf.com/latex/examples/latex-coffee-stains/qsjjwwsrmwnc}{For Coffee Lovers} - place coffee stains on \LaTeX{} documents.
\item \href{https://project-awesome.org/egeerardyn/awesome-LaTeX}{Even more resources} - an awesome list of \LaTeX{} resources.
\end{itemize}
\end{enumerate}

% Printing Index %
\printindex

\end{document}